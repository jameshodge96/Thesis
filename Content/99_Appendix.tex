
\chapter{Methodology documents}
The following participant information, consent and data management sheets are representative of the processes used for all of the studies described in this thesis. The only difference between the different methodology sheets is the content explaining the study purpose and wording alteration depending on stakeholders.

\section{Information Sheet}
\label{app:infoSheets}
\begin{figure}[htp]
    \centering
    \includegraphics[width=0.3\linewidth]{Images/logo.png}
\end{figure}
\begin{quote}
\textbf{Information sheet: Designing learning resources about dementia for people with background
in design and computer science}
\end{quote}

\textbf{This information sheet tells you about participating in a research study run by Open Lab, Newcastle University. Before you decide to take part in the study, it is important for you to understand why the research is being done and what will be involved. Please take time to read the following information carefully.
}

Please contact James Hodge if there is anything that is not clear, or if you want more information about the study:

\textbf{Tel: *******}

\textbf{Email: *****}

\textbf{Aim of this research}

The research aims are to explore designing a series of design resources to support learning and designing technology or design products for people with dementia. 

\textbf{Study summary}

You are being asked to participate in this study because you have experience of dementia. If you agree, you will participate in up to three interviews across June – July, with each, roughly lasting 45 minutes – 1 hour.  In these sessions, you will be asked to discuss the type of topics/conversations that are important for designers/developers to be considering when they are designing for people with dementia. During our interviews, the lead researcher will also be working alongside several designers/developers to understand the ways they approach designing products and the type of design procedures they use. As an output of this study, your contribution will provide a ‘starter pack’ targeting designers and developers to learn and develop more evocative experiences for people with dementia. Each interview will be recorded and transcribed. No sensitive information is required, and names and details will be kept anonymous if you choose to do so (in the consent sheet). 

\textbf{Do I have to take part?}

No, it is up to you to decide whether to take part in the study or not. If you consent to participate, we will explain what is involved and give you the information sheet to keep. You will then be asked to sign a consent form which you can discuss with family and friends if you wish. During the study, you are still free to withdraw from the study at any time for no given reason.

\textbf{What will happen to the data we collect?}

We will treat any information from this study with great care and follow the General Data Protection Regulation (GDPR). Any personal information including audio-visual recordings will be stored on a secure, password-protected computer.

\textbf{Thank you for reading this information sheet and for your interest in this study.
}

\newpage

\section{Consent Sheet}
\label{app:ConSheets}
\begin{figure}[htp]
    \centering
    \includegraphics[width=0.3\linewidth]{Images/logo.png}
\end{figure}
\begin{quote}
\textbf{Consent form: Designing learning resources about dementia for people with background
in design and computer science}
\end{quote}

\textbf{I \_\_\_\_\_\_\_\_\_\_ agree to participate} in the study and partaking in a series of workshops / interviews around exploring the type of resources and priorities that are required to create an effective set of learning documents for understand more about dementia. This research will be primarily conducted by James Hodge, at Open Lab, Newcastle University.

\begin{itemize}
    \item \textbf{The study has been explained to me and I understand it.}
    \item \textbf{I have had all of my questions answered}
    \item \textbf{I am participating voluntarily}
\end{itemize}

I understand I have the right to withdraw at any time without penalty. To withdraw, simply contact James Hodge by email / phone:

\textbf{Tel: ******}

\textbf{Email: *****}

All of \textbf{my data will be kept confidential}. With this, \textbf{your data will remain anonymous}, unless we have otherwise specified, during the project and after closure.

\textbf{Please tick the box:}
Yes     No

If yes, please choose an anonymous name for yourself (if left empty, we will
create an anonymous name for you instead)


\_\_\_\_\_\_\_\_\_\_\_\_\_\_\_\_\_

I give permission for my interactions from this study, to be recorded in written,
audio, and video recorded form, and may be used in an examined dissertation,
public talks, and any successive publications if I give permission below:

\textit{(Please select all that apply)
}\begin{itemize}
    \item I agree to the use of quotes in publication from my data
\end{itemize}
OR
\begin{itemize}
    \item I do not agree to anonymised quotes in publication of extracts from my data.
\end{itemize}


Signed: \_\_\_\_\_\_\_\_\_\_\_\_\_\_\_\_\_

\newpage
\section{Data management}
\label{app:DataManagement}
\begin{enumerate}
    \item \textbf{What data will be produced?}
\end{enumerate}	
Data will be produced through audio recording interviews and group activities in workshops. The recordings will be translated and transcribed into text files (.doc/ pdf/a). Observational notes will be handwritten by researchers and then written up to be digitised (.doc / pdf/a). In addition, still images will be taken to document the group activities and ‘day-out’ session. Some participants will take 360-degree videos (.mp4) during the process.

\begin{itemize}
    \item workshops, audio recorded, 1.5 hours each X2
    \item Interviews, audio recorded, 40 minutes each X5
    \item Workshops, Transcripts, 8 pages each X2
    \item Interviews, Transcripts, 4 pages each X5
    \item ‘day-out’, audio recorded, 3.5 hours 
    \item ‘day-out’, Transcripts, 15 pages 
    \item Observational notes, 20 pages
    \item Still Images X40
    \item 360-degree videos X8
\end{itemize}

\begin{enumerate}[resume]
\item \textbf{What metadata standards will you use?} 
\end{enumerate}
The data will be documented to record the procedure in a spreadsheet (.csv) file. This will include interview, workshop and day-out schedules. Schedules will have breakdowns of what will occur i.e. interview questions and will be anonymized completely including dates and locations. 

Metadata will be used to differentiate between the type of activity that the data was collected (e.g. interview compared focus group) and will include the description and keywords of the data. It will also contain the date the data was collected to maintain a chronological narrative. 

\begin{enumerate}[resume]
\item \textbf{How will your data be structured and stored?} 
\end{enumerate}
Structured: 

Digital data will be structured in project specific folders on Newcastle University storage (see below for further information). The digital data will also be created in ‘open’ file formats including pdf/a and tif. Additionally, the files will be separated according the data collection method.

Stored: 

All audio recordings will be immediately transferred from the audio recorder to a secure password protected computer and deleted off of the recording device and backed up on Newcastle University’s filestore. Transcription of data will be done by the research team in a private place. When transcriptions are completed they will be handled with caution, stored in the secure password protected computer in Open Lab and the Center for Research on Population and Health and the audio files will be permanently deleted. While in use all digital copies of files will be encrypted, password protected and stored securely on Newcastle University’s filestore.

Newcastle’s filestore service is hosted across two data centres, equipped with fire detection, suppression equipment, and secure audited access procedures. In addition, it operates \href{https://services.ncl.ac.uk/itservice/core-services/filestore/shadowcopy/}{`Shadow Copies'}, which are taken four times daily. An incremental copy to backup tape is taken nightly, and a full copy monthly. Backups are kept for ninety days. Inactive tapes are stored in on-campus fireproof safes. The Summary of Technical Information Security for Information Systems and Services provides more detail.  

All physical materials produced (i.e. sketches) will be stored in a secure location at the CRPH and Open Lab. 

Still images collected form the design workshops will be digitally stored in the afore mentioned manner however faces will be blurred before storage and the original images will be completely deleted off of the image taking device and the computer used to blur the faces.

This study is Engineering and Physical Sciences Council funded, therefore anonymised data that underpins publications are required to be archived and kept for at least ten years. For publications created during this project the anonymised data will be will be stored at Newcastle University’s \href{https://data.ncl.ac.uk}{data catalogue}. The data catalogue will create a persistent link (Digital Object Identifier) for inclusion in the publication that will direct the reader to where they can access the data. It is important to note that all the data stored in this manner will be anonymised and only data that the participant has agreed to be shared in this format will be included. Not all research data will be shared and where data is to be restricted the researcher will provide a reason why (i.e. sensitivity of data). 

\begin{enumerate}[resume]
\item \textbf{How will the data be shared during and after the project?} 
\end{enumerate}
During the project data will be shared among the research team (Dr. Kellie Morrissey, Dr. Kyle Montague) through Open Lab’s office365 server that is both secure and private. 

Anonymised data supporting a publication may be shared with other researchers under EPSRC policies and outlined above. 



\newpage
\section{Thematic Analysis Example}
\label{app:TA}

\begin{figure}[htp]
    \centering
    \includegraphics[width=0.6\linewidth]{Images/Appendix/Analysis/Codebook_Overview.png}
    \caption{NVivo 12 hierarchy chart of 'nodes' (codes \& themes) for data found in \ref{EthicsChapter}}
    \label{fig:App:TA-Codebook}
\end{figure}

\begin{figure}[htp]
    \centering
    \includegraphics[width=0.6\linewidth]{Images/Appendix/Analysis/ParticipantTranscriptCoding.png}
    \caption{NVivo 12 code exploration diagram of one participant's interview transcript}
    \label{fig:App:TA-TranscriptCoding}
\end{figure}

\chapter{DemVR}

\section{Participants' final ideas}
\label{sec:EventIdeas}

\begin{figure}[htbp]
\begin{subfigure}[t]{0.3\textwidth}
    \includegraphics[width=\linewidth]{Images/DemVR/GardenLife.png}
\caption{Garden Life}
\label{fig:gardenLife}
\end{subfigure}\hfill
\begin{subfigure}[t]{0.3\textwidth}
  \includegraphics[width=\linewidth]{Images/DemVR/ChatterBench.png}
\caption{Chatter Bench}
\label{fig:ChatterBench}
\end{subfigure}\hfill
\begin{subfigure}[t]{0.3\textwidth}
    \includegraphics[width=\linewidth]{Images/DemVR/AugmentedWorld.png}
\caption{Augmented World}
\label{fig:AugmentedWorld}
\end{subfigure}

\begin{subfigure}[t]{0.3\textwidth}
    \includegraphics[width=\linewidth]{Images/DemVR/VRHallucinate.png}
\caption{VRHallucinate}
\label{fig:VRHallucinate}
\end{subfigure}\hfill
\begin{subfigure}[t]{0.3\textwidth}
    \includegraphics[width=\linewidth]{Images/DemVR/LookingVRBack.png}
\caption{Looking VR Back}
\label{fig:LookingVRBack}
\end{subfigure}\hfill
\begin{subfigure}[t]{0.3\textwidth}
    \includegraphics[width=\textwidth]{Images/DemVR/MindfulForest.png}
\caption{Mindful Forest}
\label{fig:MindfulForest}
\end{subfigure}

\begin{subfigure}[t]{0.3\textwidth}
    \includegraphics[width=\linewidth]{Images/DemVR/SensoryTides.png}
\caption{Sensory Tide}
\label{fig:SensoryTide}
\end{subfigure}\hfill
\begin{subfigure}[t]{0.3\textwidth}
    \includegraphics[width=\linewidth]{Images/DemVR/WorldShare.png}
\caption{WorldShare}
\label{fig:WorldShare}
\end{subfigure}\hfill
\begin{subfigure}[t]{0.3\textwidth}
    \includegraphics[width=\linewidth]{Images/DemVR/VRMotion.png}
\caption{VRMotion}
\label{fig:VRMotion}
\end{subfigure}
\caption{DemVR final prototype ideas}
\label{fig:DemVRFinalIdeas}
\end{figure}
From the final nine ideas, each team developed a prototype of their final idea alongside their ten minute presentation. In this subsection, I briefly breakdown each teams' proposed ideas and their final ideas:

\subsubsection{a. Garden Life (seven undergraduate computing students}
\label{sec:gardenLife}
\textbf{Proposed Idea:} Garden Life's proposed idea originated from Ideaboard where their idea would be to \textit{"create a journey through the story of your life using media that links memories with locations."}

\textbf{Final Idea:} In the teams' final idea, they created a VR garden and dog companion for those living in isolation without access to either in real life. The team had considered ways for sharing experiences by carers assisting the growth of a virtual garden through the extension of a tablet while the person living with dementia used VR. Furthermore, the team added multiplayer aspects to the experience allowing family and friends to virtually join the user in their own personalised garden with their virtual dog. 

\subsubsection{b. Chatter Bench (two designer / researchers) - Won 2nd Prize}
\label{sec:chatterbench}
\textbf{Proposed idea:} A chat-based VR experience that will consider the importance of sensitivity and aesthetic design. This idea was not developed on Ideaboard, but came from conversations between the two team members.

\textbf{Final Idea:} In teams' final idea, they developed a shared VR experience in Unity game engine where two users could 'sit' and talk on a virtual bench. Either the care partner, or person living with dementia could select from an array of different 360-degree environments where the two users could talk into a microphone and hear one another.

\subsubsection{c. Augmented World (six undergraduate computing students}
\label{sec:augmentedWorld}
\textbf{Proposed Idea:} A bespoke chronological AR timeline connected to experiences of the users past and family. The proposed idea was developed on Ideaboard by two of the six members who originally focused their idea on themes of reminiscence. 

\textbf{Final Idea:} An AR app that 'enhances' environmental objects to facilitate meaningful social interaction by connecting virtual objects to overlay on the real-life object. For example, a user can scan a picture of their family in the app which will then overlay virtual text or videos onto the picture.

\subsubsection{d. VRHallucinate (six members from developer, researcher, and UX backgrounds}
\label{sec:VRHallucinate}
\textbf{Proposed Idea:} As the team joined the event late and missed the pre-engagement phase, the team decided to develop a gamified experience to visualise hallucination-like effects to raise awareness of potential cognitive deficits that one may get with a diagnosis of dementia.

\textbf{Final Idea:} Similar to their proposed idea, they designed the hallucination game but targeted elements of ways to educate family, friends and the public surrounding the potential challenges of living with dementia.

\subsubsection{e. Looking VR Back (four members from marketing, development, biomedical backgrounds}
\label{sec:VRBack}

\textbf{Proposed Idea:} Using sounds, scents, and colours as a way to support reminiscence - this was proposed on Ideaboard by the same team.

\textbf{Final Idea:} Same idea but used a personalised example relating to Newcastle football game in 1969 as a way to take people living with dementia back to the particular experience through scents, sounds and 60-70's VR room aesthetic. 

\subsubsection{f. Mindful Forest (two undergraduate computing students}
\label{sec:mindfulForest}

\textbf{Proposed Idea:} A fantasy shared experience where families can add videos to trigger past memories. This idea was developed at the event as the team did not engage with Ideaboard.

\textbf{Final Idea:} A forest-like environment with gentle music and scenery. Families and friends can add videos to help with memory stimulation.

\subsubsection{g. Sensory Tide (six members from developer, researcher, and film backgrounds) - Won 1st Prize}
\label{sec:senosryTide}

\textbf{Proposed Idea:} The team combined several participants during the team formation, where multiple Ideaboard ideas came together. At the initial stage of developing their idea, the team had ideas of: replacing VR headsets with full-dome projections, themes of focusing on designing for the moment, as oppose to improving cognitive deficits, and designing reminiscence tools to increase Independence doing tasks.

\textbf{Final Idea:} The team developed an adapted version of a headset to resemble a beach telescope that was easily accessible. Additionally, the team created the VR beach experience to support multi-sensory needs for example, the user could smell seaweed in the room, a heated fan attached to the wall and sand under the feed of the user. 

\subsubsection{h. World Share (three filmmakers}
\label{sec:WorldShare}

\textbf{Proposed Idea:} Developed at the event, the team's initial idea was a tablet-based app to allow people living with dementia to 'travel' across the world and take part in tours and explore other scenic outdoor locations.

\textbf{Final Idea:} The final idea consisting of the following: a tablet that acts as a controller for interaction, and a headset to be used by the person living with dementia. The person with dementia would then request an event or a family recorded experience which the care partner would navigate to on the tablet and send the video/experience to the VR headset.

\subsubsection{i. VRMotion (for members from developer and researcher backgrounds)}
\label{VRMotion}
\textbf{Proposed Idea:} Influenced by their prior work in research, the team's initial idea was care home oriented where the VR experience would be a celebration of abilities of the individual instead of trying to bridge the abilities they have lost. 

\textbf{Final Idea:} Building from their proposed idea, they developed a shared virtual world where people with dementia can take part in group activities such as songs to sing along, guess the place, and solve the riddle - that was inspired by the work of Foley et al. \citep{foley_printer_2019}


\newpage
\section{DemVR branding}
\label{app:Branding}
In this section, I provide several images of the branding and ways we used our branding for recruitment purposes.


\begin{figure}[htp]
    \centering
    \includegraphics[width=0.8\linewidth]{Images/Appendix/DemVR appendix/DemVR Flyer.png}
    \caption{Leaflets placed across Newcastle city.}
    \label{fig:App:Leaflets}
\end{figure}

\begin{figure}[htp]
    \centering
    \includegraphics[width=0.8\linewidth]{Images/Appendix/DemVR appendix/Lanyard.png}
    \caption{Bespoke DemVR lanyards to make people feel part of the event during the hackathon.}
    \label{fig:App:Lanyard}
\end{figure}

\begin{figure}[htp]
    \centering
    \includegraphics[width=0.8\linewidth]{Images/Appendix/DemVR appendix/Banner.png}
    \caption{DemVR TV banner shared on Newcastle University TV's on campus.}
    \label{fig:App:TV-Banner}
\end{figure}

\begin{figure}[htp]
    \centering
\includegraphics[width=0.8\linewidth]{Images/Appendix/DemVR appendix/DemVR_Twitter.png}
    \caption{Twitter account to build-up hype around the event and live tweet during the 2-day event. Check out $@$DemVR\_UK. }
    \label{fig:App:Twitter}
\end{figure}

\newpage
\section{DemVR Event programme}
\label{app:DemVREvent}
Below, I provide the detailed event programme that I used for the advertisement and schedule material. In here you will find the event overview and the event schedule. 

\subsection{Event Overview}
\label{app:DemVR:Overview}
A big focus in dementia and technology research has been to tackle the cognitive deficits that often accompany the condition. Virtual reality has been used in the assessment and rehabilitation of cognitive processes in dementia since the 1990s, and more recently it's been used to deliver exergames. These developments are no doubt exciting - however, the potential for virtual reality as an expressive and creative medium to allow people with dementia to experience new, exciting, stimulating and potentially therapeutic environments, entirely separate from the stress of cognitive assessment, should also be addressed.

This event, organised by Newcastle University and held in central Newcastle, will provide an environment for innovative and creative ideas to emerge surrounding how we might create enriching shared experiences for people living with dementia. At the beginning of this two-day hackathon, we will help to `matchmake' participants and aid in the formation of multidisciplinary teams (ready-formed teams are also welcome!). Saturday morning will see the event introduced by keynotes from industry, research, and practice experts. Teams will be provided with creative material and qualitative data gathered from people living with dementia, as well as research from experts in the area of technology and dementia, in order to inspire them to create assets and environments that might enrich the shared experiences of those affected by the condition. After a solid 24 hours+ where participants are free to ideate, design, hack and make to their hearts' content, teams will have the opportunity to demonstrate, and present their reasoning behind their design choices to a team of expert judges in dementia care and research, who will award a monetary prize to the winning team.   

\textbf{First Prize: £1000}

\textbf{Second Prize: £500}



\subsection{Event schedule}
\label{app:DemVR:EventSchedule}
\begin{figure}[htp]
    \centering
    \includegraphics[width=0.8\linewidth]{Images/Appendix/DemVR appendix/Schedule.png}
    \caption{Schedule of DemVR}
    \label{fig:App:DemVRSchedule}
\end{figure}

% Please add the following required packages to your document preamble:
% \usepackage{graphicx}
\begin{table}[htbp]
\caption{DemVR two-day detailed schedule}
\label{tab:DemVR-Detailed schedule}
\resizebox{\columnwidth}{!}{%
\begin{tabular}{l|ll}
\textbf{\begin{tabular}[c]{@{}l@{}}Pre-hackathon\\ (In-person),\\ Friday, April 5, 2019\end{tabular}} &
  \begin{tabular}[c]{@{}l@{}}TEAM FORMATION\\ \\ Team formation will be held at Newcastle\\ \\ Pre-registration will be held on Friday before the hackathon starts to bring \\ participants together for team formation. We will also do an overview of the ideas \\ created on ideaboard. Feel free to choose one of these ideas or start from scratch \\ and create your own. Coming to the Friday event will help designers and \\ developers understand what is expected over the weekend and to get to know \\ everyone who is taking part in the hackathon.\\  \\ Recommended team size: 3-6\\ \\ 6pm: Meet and Greet / registration \\ 7pm: Team Formation \\ 8pm: Food / Drinks in town (at own cost)\end{tabular} &
   \\ \cline{1-2} \\
\textbf{\begin{tabular}[c]{@{}l@{}}Day 1\\ Hackathon,\\ Saturday, April 6, 2019\end{tabular}} &
  \begin{tabular}[c]{@{}l@{}}HACKATHON\\ The hackathon will begin Saturday morning with all teams formed \\ (any teams not organised, will be organised on the morning). The \\ day will start out with three keynote speakers talking about their \\ experiences about designing technology with people living with \\ dementia. Afterwards, teams will have free rein to work on their \\ designs for over 24 hours. \\ \\ INTRODUCTIONS \\ 8am - 9am: Breakfast \& registration (if you haven't registered on Friday) \\ 9am - 10am: Keynote Presentations \\ \\ HACKING STARTS \\ 10am – 11am: Start working on your idea\\ 10:30am – 11am: WhatsApp reflection period\\ 11am – 11:15am: tea, coffee, and biscuits\\ 11:15am – 12pm: Online Q\&A with Howard\\ 12pm – 1pm: WhatsApp reflection period\\ 1pm – 2pm: Lunch to be served\\ 2pm – 2:30pm: Pitch your idea to the room + Get feedback from Experts \\ 4:30pm - 5pm: Pitch Back \\ 6:30pm: Dinner\end{tabular} &
   \\ \cline{1-2} \\
\textbf{\begin{tabular}[c]{@{}l@{}}Day 2,\\ Judging,\\ Sunday, April 7, 2019\end{tabular}} &
  \begin{tabular}[c]{@{}l@{}}JUDGING\\ Sunday morning will give teams time to finalise their ideas and \\ set up for judging. Each team will present their idea and creations \\ to a panel of judges comprised of domain experts, designers. \\ Prizes will be awarded based on creativity, originality, and \\ project demonstration. \\  \\ 9am - 10am: Breakfast \\ 9am - 10am: WhatsApp reflection period\\ 11am – 11:30am: Tea, coffee, and biscuits\\ 12pm – 1pm: Lunch to be served\\ 1pm – 2pm: WhatsApp reflection period \\ 2:15pm – 4pm: Team presentations\\ 4pm – 4:20pm: Judges scoring \& WhatsApp reflection period*\\ 4:20pm 5pm: Awards and closing\end{tabular} &
  
\end{tabular}%
}
\end{table}

\chapter{Everyday interactions in dementia}

\section{Ethics Semi-structured interview}
\label{app:EthicInterview}
The following questions were used for the 22 interviews with HCI researchers.

\begin{enumerate}

   \item \textbf{What work have you done in a dementia context?}
\begin{enumerate}
        \item Can you elaborate on our expertise and experience?
\end{enumerate}

    \item \textbf{What do you think about the role of the individual researcher vs. the institutional responses to protecting researchers}
    \begin{enumerate}
        \item Can you tell me a bit about your interactions with ethical review processes at your institution?
        \item Do you think that they are effective at protecting participants? Researchers?
        \item How do you think they change when you talk about technology as being a part of your work?
    \end{enumerate}

    \item \textbf{How do you think implementing technology into technology has added to the ethical complexities of working with people living with dementia?}
    \begin{enumerate}
        \item Do you think that seeking to innovate technologically has made working with people with dementia more ethically complex? If so, how?
        \item How do you think such work is perceived by the media? The larger research community?
        \item Do you think people with dementia and their carers/families want technology in their lives?
    \end{enumerate}

    \item \textbf{Have you felt a power imbalance of sorts between you as the researcher or significant role changes between the carer and the person with dementia?}
    \begin{enumerate}
        \item Can you tell me a bit about the sorts of relationships you’ve had in dementia contexts?
        \item (if there are many) What do you feel was the most interesting or striking relationship?
        \item In particular, I’m interested in a sense of power balance or imbalance, or instances where your role changed.
    \end{enumerate}

    \item \textbf{Have you had instances where you’ve become part of a community/friends with your participants? Do you feel there is a line? Are the steps you’d suggest to other researchers to follow or have in the back of their heads?}
    \begin{enumerate}
        \item Have you had instances where you’ve become part of a community or friends with your participants?
        \item Do you feel there is a line that should or shouldn’t be crossed?
        \item Are the steps you’d suggest to other researchers to follow or have in the back of their heads if this is something that’s happening for them?
    \end{enumerate}

    \item \textbf{How can we make sure our efforts have longevity?}
    \begin{enumerate}
        \item Do you feel like your research, or HCI research into dementia in general, has longevity?
        \item What are some ways we can extend our longevity?
        \item Are there instances where longevity isn’t as important as we might think?
    \end{enumerate}

    \item \textbf{Questioning reliability of technology vs short and fast iteration of technology}
    \begin{enumerate}
        \item How highly do you value the reliability of the technologies you create?
        \item Do you feel like agile design environments or processes are suitable to the creation of technologies for people with dementia?
        \item Can you describe some of the design processes you’ve undertaken with people with dementia in the past, from a methods point of view?
    \end{enumerate}

    \item D\textbf{esigning for an exit strategy. It needs to be more than just a debrief document.}
    \begin{enumerate}
        \item Negotiating a process of leaving can be difficult when you’ve come to form attachments to your participants. Can you tell me a bit about how you’ve left research contexts in the past?
        \item What about leaving technologies or designs behind? What sort of responsibilities do we have then?
    \end{enumerate}

    \item \textbf{Anonymisation is a key premise of how we conduct ethical research to protect the privacy and integrity of those we’re working with. While we typically default to this, do you think thats a good idea? Are we hindering acknowledgment of those we’re working with and not crediting their creative and intellectual work.}
    \begin{enumerate}
        \item How do you feel about anonymisation in design research with dementia?
        \item Is it always a good thing, or is it more ethical to identify participants’ achievements and labour?
        \item What would an updated best practice look like here?
    \end{enumerate}

    \item \textbf{How useful are examination boards? How much do they know about the research they are overseeing and do they see the clear benefits of the research we’re doing?}
    \begin{enumerate}
        \item Can you tell me about your experiences with ethical review boards?
        \item How much do you feel they know about the research they are passing judgement on?
        \item Do they always see the benefits of our research?
        \item When we are working in an ethically dynamic environment to what extent are ethical review boards aware of the changing environment or restrict ground breaking research in this area?
    \end{enumerate}
 
    \item \textbf{At what point should we publish? Should we place quality over quantity in publications? How can we challenge the problems of expectations from participants from previous work?}
\end{enumerate}

\chapter{D3 Toolkit}
Here, I provide examples of the type of design activities participants engaged in across the iterative workshops.

\section{Workshop One}
\label{D3:W1}
\begin{figure}[htp]
    \centering
    \includegraphics[width=0.6\linewidth]{Images/Appendix/D3Toolkit/Workshop-One-Scenarios.jpg}
    \caption{A set of design fiction scenarios to provoke questions around early phases of design and development, sensitive conversations within teams, and reflecting on mistakes.}
    \label{fig:App:W1-Scenarios}
\end{figure}

\begin{figure}[htp]
    \centering
    \includegraphics[width=0.6\linewidth]{Images/Appendix/D3Toolkit/Workshop-One.jpg}
    \caption{Final activity to explore the type of toolkit/resources the participants would design in response to the ethical issues in the previous activity.}
    \label{fig:App:W1-Final}
\end{figure}

\newpage
\section{Affinity Diagramming}
\label{app:AD}
\begin{figure}[htp]
    \centering
    \includegraphics[width=0.8\linewidth]{Images/Appendix/D3Toolkit/Affinity-Diagram.jpg}
    
    \caption{Affinity diagram of stage one and stage two data to provide a set of design rationales.}
    \label{fig:App:AffinityDiagram}
\end{figure}

\newpage
\section{Workshop Two}
\label{D3:W2}
\begin{figure}[htp]
    \centering
    \includegraphics[width=0.8\linewidth]{Images/Appendix/D3Toolkit/Workshop2-Example.jpg}
    \caption{Participants explored ten ideas which emerged via the process of affinity diagramming that was split into five themes - storytelling, seeking partnership, reflection, directing the conversation, and community-driven toolkits.}
    \label{fig:App:W2-Example}
\end{figure}

\newpage
\section{Workshop Three}
\label{D3:W3}
\begin{figure}[htp]
    \centering
    \includegraphics[width=0.6\linewidth]{Images/Appendix/D3Toolkit/Workshop3-Overall.jpg}
    \caption{Prototype toolkit that was used in workshop three}
    \label{fig:App:W3}
\end{figure}

\begin{figure}[htp]
    \centering
    \includegraphics[width=0.6\linewidth]{Images/Appendix/D3Toolkit/Workshop3Creation.jpg}
    \caption{Example of the automated suggestion tools described to fit the developers workflows}
    \label{fig:App:W3-AutomtedTools}
\end{figure}


\chapter{Data chapter summaries}
\label{Appendix:ResearchQuestionsMapped}
Here, I provide mapping of each data chapter to the research questions set out at the start of the thesis.

\begin{table}[htp]
    \centering
    \begin{tabular}{p{0.15\linewidth} | p{0.8\linewidth}}
      \multicolumn{2}{p{0.95\linewidth}}{\textbf{Research question one: How can we use participatory design approaches to provide meaningful and engaging experiences for people with dementia?}} 
      \\   \hline
      Chapter Four &
      \begin{itemize}
          \item The use of workshops often relied on the care partner's support as the workshop relied more on verbal communication. Alternatively, using walking interviews (described in section \ref{PD:Interviews}), provided a less stressful place, and focused on the moment rather than solely relying on the participant's memory.
      \end{itemize}
 \\   \hline
    Chapter Five &
    \begin{itemize}
          \item Teams would rely on stakeholders' experiences of dementia to construct their understanding of dementia. Those who did not engage with other stakeholders drew on Howard's Q\&A to represent who they were designing 'for' (seen in \ref{ThemeTwo:AbsentUser}). Through the hackathon structure, participants adopted person-centred ideas to dementia by engaging with facilitators, inspiration packs and resources curated by dementia organisations (seen in \ref{ThemeThree:SenseofDementia})

      \end{itemize}
       \\ \hline
        Chapter Seven &
      \begin{itemize}
          \item Aware of the recruitment challenges I faced in chapter five, I significantly altered my recruitment process in chapter seven to ensure that people with dementia would be incentivised to participate in the study. I made sure to build rapport with those recruited through initial Zoom calls to get to know one another and build a list of requirements to accommodate the participant's needs to ensure they are engaged in the research.

          \item  Designers and developers also felt being accountable for technology creation needs further attention where accessibility and inclusivity of websites need to be more transparent to encourage companies to support the involvement of people with dementia.
      \end{itemize}
    \end{tabular}
    \label{chaptersRQ1}
\end{table}

\pagebreak

\begin{table}[htp]
    \centering
    \begin{tabular}{p{0.15\linewidth} | p{0.8\linewidth}}
    \multicolumn{2}{p{0.95\linewidth}}{\textbf{Research question two: What are the ethical implications for people with dementia to participate in HCI research?}} 
    \\ \hline
     Chapter Four &
      \begin{itemize}
         \item The first study designed a set of VR environments based on locations that resonated with the participant's childhood, and reminiscence can provide positive experiences, for some, it may cause frustrations when they cannot remember specific memories \citep{lazar_critical_2017}.
         
        \item When designing technology, it is essential to note that they are additional challenges relating to robustness, longevity and leaving technology with participants when the project ends. Beyond leaving technology as a form of impact in our research, my relationships with the family held significance to the family, demonstrating the personalisation, recognition, and meaning of participation was an invaluable part of the study. 
        
        \item In study two, researching under a culture of ethical `protectionism' caused significant tensions where the participants could not provide verbal consent at the start, resulting in their inability to contribute to the study. From highlighting several ethical implications within the work, I stress that further work must concern ethical review boards and understanding how other researchers navigate the ongoing ethical complexities in dementia.

      \end{itemize}

       \\ \hline
        Chapter Five &
    \begin{itemize}
          \item Teams highlighted the difficulties in constructing who their users might be - emphasising the importance of engaging with people with dementia and the knock-on effects this had on the final ideas (seen in \ref{ThemeTwo:AbsentUser} and \ref{ThemeThree:SenseofDementia}).

          \item Some teams prioritised ways to mitigate the discomfort of VR by considering ways to navigate, use of language around `VR', and how to sensitively represent the VR environment (seen in \ref{ThemeThree:SubThemeOne}). Teams' ideas presented challenges and considerations needed for the shared VR experience (seen in \ref{ThemeThree:Subthemetwo}).

      \end{itemize}
       \\ \hline
       Chapter Six &
    \begin{itemize}
          \item Within this chapter, participants also provided insights into what it means to have an impact from a research perspective. For some, it is the technology that's created. However, this raised similar concerns to chapter four, where I had anxiety and frustrations with technology not being robust enough to be left behind. As researchers in this chapter recall, researchers will often not have enough time or resources to develop robust technologies.

          \item The analysis revealed tensions between institutional ethical practices put in place by ethical review boards. While they are in place to protect participants, researchers reported on varying cultural and disciplinary approaches to dementia research. This chapter reveals that researchers find tensions from ERBs when attempting to design participatory approaches in their work with people with dementia.

      \end{itemize}
    \end{tabular}
    \label{chapterRQ2}
\end{table}


\begin{table}[htp]
    \centering
    \begin{tabular}{p{0.15\linewidth} | p{0.8\linewidth}}
    \multicolumn{2}{p{0.95\linewidth}}{\textbf{Research question three: What are the competing interests and expectations to support meaningful dialogue in dementia design research when involving multiple stakeholders - such as people with dementia, developers, designers and researchers?}} \\ \hline
      Chapter Four &
          \begin{itemize}
          \item I reported on designing media experiences with not only the person with dementia but also their family members and friends. By spending time with the care partners and family members and focusing on memorable and pleasurable activities, the overall technologies I built had various interests and interactions that the ecology of care desired. Taking such an approach looked towards seeing each member as whole persons rather than defining them by their roles as a care receiver/giver.
      \end{itemize}
   \\ \hline
    Chapter Five &
    \begin{itemize}
          \item While participants' motivation for taking part ranged from prize money, learning about dementia, and personal experiences, the incentives for people with dementia gave no real encouragement for taking part (seen in \ref{ThemeOne:subthemeOne}). 

          \item While chapter four demonstrated people with dementia having an appeal in participating in the research as the days were designed around their interests and needs, the hackathon overlooked the incentives for people with dementia and their care partners. As described in the commitments, if I had worked with community members, I may have provided alternatives to hackathons that felt more appropriate to people with dementia. For example, I could have taken the co-design days out of chapter four and facilitated conversations between designers, developers and people with dementia through a similar walking interview process.
      \end{itemize}
  \\ \hline
    Chapter Six &
    \begin{itemize}
          \item A key finding was how researchers acknowledge people with dementia for their contribution. For some, that might mean being named as co-authors on academic papers. However, for others, it is just appreciated by the research team for the person with dementia's time and commitment to helping, which resonates with the families from chapter four.

        \item To support engaging and participatory approaches with people with dementia, researchers expected an improved relationship with ethical review boards to collaborate and seek support for research processes and priorities. Similar to my own experiences in chapter four, researchers commented on the importance of self-care when conducting work within sensitive settings. 
      \end{itemize}
    
    \end{tabular}
\end{table}
\clearpage
\begin{table}[ht]
    \begin{tabular}{p{0.15\linewidth} | p{0.8\linewidth}}
    \\ \hline
    Chapter Seven &
          \begin{itemize}
          \item Like chapter six, people with dementia reported wanting to be acknowledged for their contribution to technology design. People with dementia reiterated that this is not about money but researchers acknowledging the time through relationship building. However, when it came to companies or organisations inviting people with dementia to workshops or interview activities, the person with dementia was expected to be paid as a consultant. 
          \item For designers and developers, the fundamental interest was that conversational and critical thinking activities must fit into their current workflows. Additionally, to provide designers and developers the confidence to reach out and talk to people with dementia, we must provide resources to help destigmatise and support conversations. For instance, this chapter highlighted resources to show various types of dementia experiences alongside conversational cards and prompts for 'ice-breakers' when first reaching out to someone with dementia.

      \end{itemize}
    \\ 
    
    \end{tabular}
    \caption{Findings mapped to research questions}
\end{table}

