
\chapter{Documents}

\section{Information Sheets}
\label{app:infoSheets}

\section{Consent Sheets}
\label{app:ConSheets}
- Probably don't need exmaples of each consent / information sheet from each studies. Probably just take the one I used for the media experiences project?

\section{Data management}
\label{app:DataManagement}
- Data Management PDF from the first case study

\section{Thematic Analysis Example}
\label{app:TA}
- Take this from NVivo - print out the codebook

\section{DemVR Schedule}
\label{app:DemVRSchedule}
DemVR schedule - of the two days

\section{DemVR Recruitment Advertisement}
\label{app:DemVRRecruitment}
- the different advertisement we used / flyers / posters / website

\section{Ethics Semi-structured interview}
\label{app:EthicInterview}
The following questions were used for the 22 interviews with HCI researchers seen in chapter \ref{EthicsChapter}.

\begin{enumerate}

   \item \textbf{What work have you done in a dementia context?}
\begin{enumerate}
        \item Can you elaborate on our expertise and experience?
\end{enumerate}

    \item \textbf{What do you think about the role of the individual researcher vs. the institutional responses to protecting researchers}
    \begin{enumerate}
        \item Can you tell me a bit about your interactions with ethical review processes at your institution?
        \item Do you think that they are effective at protecting participants? Researchers?
        \item How do you think they change when you talk about technology as being a part of your work?
    \end{enumerate}

    \item \textbf{How do you think implementing technology into technology has added to the ethical complexities of working with people living with dementia?}
    \begin{enumerate}
        \item Do you think that seeking to innovate technologically has made working with people with dementia more ethically complex? If so, how?
        \item How do you think such work is perceived by the media? The larger research community?
        \item Do you think people with dementia and their carers/families want technology in their lives?
    \end{enumerate}

    \item \textbf{Have you felt a power imbalance of sorts between you as the researcher or significant role changes between the carer and the person with dementia?}
    \begin{enumerate}
        \item Can you tell me a bit about the sorts of relationships you’ve had in dementia contexts?
        \item (if there are many) What do you feel was the most interesting or striking relationship?
        \item In particular, I’m interested in a sense of power balance or imbalance, or instances where your role changed.
    \end{enumerate}

    \item \textbf{Have you had instances where you’ve become part of a community/friends with your participants? Do you feel there is a line? Are the steps you’d suggest to other researchers to follow or have in the back of their heads?}
    \begin{enumerate}
        \item Have you had instances where you’ve become part of a community or friends with your participants?
        \item Do you feel there is a line that should or shouldn’t be crossed?
        \item Are the steps you’d suggest to other researchers to follow or have in the back of their heads if this is something that’s happening for them?
    \end{enumerate}

    \item \textbf{How can we make sure our efforts have longevity?}
    \begin{enumerate}
        \item Do you feel like your research, or HCI research into dementia in general, has longevity?
        \item What are some ways we can extend our longevity?
        \item Are there instances where longevity isn’t as important as we might think?
    \end{enumerate}

    \item \textbf{Questioning reliability of technology vs short and fast iteration of technology}
    \begin{enumerate}
        \item How highly do you value the reliability of the technologies you create?
        \item Do you feel like agile design environments or processes are suitable to the creation of technologies for people with dementia?
        \item Can you describe some of the design processes you’ve undertaken with people with dementia in the past, from a methods point of view?
    \end{enumerate}

    \item D\textbf{esigning for an exit strategy. It needs to be more than just a debrief document.}
    \begin{enumerate}
        \item Negotiating a process of leaving can be difficult when you’ve come to form attachments to your participants. Can you tell me a bit about how you’ve left research contexts in the past?
        \item What about leaving technologies or designs behind? What sort of responsibilities do we have then?
    \end{enumerate}

    \item \textbf{Anonymisation is a key premise of how we conduct ethical research to protect the privacy and integrity of those we’re working with. While we typically default to this, do you think thats a good idea? Are we hindering acknowledgment of those we’re working with and not crediting their creative and intellectual work.}
    \begin{enumerate}
        \item How do you feel about anonymisation in design research with dementia?
        \item Is it always a good thing, or is it more ethical to identify participants’ achievements and labour?
        \item What would an updated best practice look like here?
    \end{enumerate}

    \item \textbf{How useful are examination boards? How much do they know about the research they are overseeing and do they see the clear benefits of the research we’re doing?}
    \begin{enumerate}
        \item Can you tell me about your experiences with ethical review boards?
        \item How much do you feel they know about the research they are passing judgement on?
        \item Do they always see the benefits of our research?
        \item When we are working in an ethically dynamic environment to what extent are ethical review boards aware of the changing environment or restrict ground breaking research in this area?
    \end{enumerate}
 
    \item \textbf{At what point should we publish? Should we place quality over quantity in publications? How can we challenge the problems of expectations from participants from previous work?}
\end{enumerate}

\section{Affinity Diagramming}
\label{app:AD}
- Miro images of the affinity diagramming in the toolkit paper

\section{Workshop materials}
\label{app:ToolkitMaterials}
- Pull in images of the different workout materials from the workshops I made