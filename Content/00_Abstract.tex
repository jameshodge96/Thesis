% ************************** Thesis Abstract *****************************
% Use `abstract' as an option in the document class to print only the titlepage and the abstract.
\begin{abstract}
In recent years, Human-Computer Interaction (HCI) researchers have recognised the importance of incorporating stakeholders to engage meaningfully with people with dementia. When done effectively, involving stakeholders such as family members, designers and developers in research and design activities, provides opportunities for a greater understanding of technology's role in dementia. However, studies that explicitly offer guidance on integrating diverse perspectives in dementia-HCI are lacking. By bridging communication between stakeholders, their views and opinions might build a more representative understanding of dementia, stressing inclusive design and participant empowerment. 

This thesis explores the experiences and perspectives of stakeholders commonly implicated but perhaps not included in design processes in designing with and for people with dementia, intending to broaden the conversation surrounding technology design with and for people with dementia. First, I provide a reflective account of adapting participatory approaches to accommodate families with dementia to design bespoke media experiences. From the lessons learned, I highlight the complexities of involving people with dementia and question how collaboration might fit into developer/designer practices. Following, I ran two studies exploring: a) the types of ethical concerns dementia-HCI researchers face in everyday interactions; b) exploring designer/developer engagements in a hackathon on technology for people with dementia. These studies illustrate that facilitating collaboration between stakeholders requires careful consideration that must articulate the interest and priorities of the individual stakeholders. Further, the thesis explores the type of resources developers/designers require to design with people with dementia and investigates how people with dementia envision participation with those implicated in the design process. 

The primary contributions are guidance for supporting collaborative design events, ethical practices in socially-orientated research, and methods for supporting dialogue between designers and people with dementia. Fundamentally, this thesis reimagines the role of participation between people with dementia and stakeholders to move towards more inclusive design spaces that support mutuality in the co-creation of new technologies and systems.

\end{abstract}
