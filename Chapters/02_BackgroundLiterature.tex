\chapter{Background Literature}
\label{BackgroundLit}


\section{Dementia and HCI}
\label{BL:DementiaHCI}
Over the last 40 years, researchers and care practitioners have seen our understanding of dementia evolve to position and represent the person with dementia in several ways. What was once a bio-medical stance has gradually moved towards one that considers the socio-political and individual experiences of dementia (Bellass et al., 2019). In the early years of dementia research, the biomedical view, dementia was viewed through its neurodegenerative condition, emphasising the person's abilities, memory, judgement and communication. Recognising these characteristics of dementia led to an awareness of strain on care partners towards policymakers, improvements in identifying and diagnosing dementia, and the development of medication to reduce symptoms of dementia (Vernooij-Dassen et al., 2021). Moreover, given that there are still no known treatments to stop the progression of dementia, there has been significant work and research funding into understanding the causes of the neurodegenerative condition.

However, several negative consequences arose from such a biomedical lens that has had significant social ramifications on people with dementia. First, the biomedical view promotes a view of 'loss of self' (Ryan et al., 2009). Early work by Cohen and Eisdorfer believed that people living with dementia "must eventually come to terms with…the complete loss of self" (Cohen & Eisdorfer, 1986). Views of dementia that see it as a state of deficiency, often place the person living with dementia as a passive "patient" where the condition is looked at being treated rather than understood. If acted upon, design and research do very little to aid the agency or the need for a continued sense of purpose and belonging (Hampson & Morris, 2016). Second, as dementia progresses, it often adds conflict between their surroundings as they become unfamiliar, and in turn, causes difficulties coexisting in places with others, such as a family home or a workplace (Au et al., 2009; Langdon et al., 2007). Finally, As we live in a society that places great value on cognitive ability, many believe that people with dementia are poor at social contact, which then prohibits many from interacting with people with dementia at all (Killick and Allan, 2001).

From the early 90's, researchers began to contest the limitations of the biomedical stance by highlighting that the quality of life of the person with dementia is determined not only by neuropathology but also by how they are perceived in society, personal history, and interactions and desires. (Personhood in dementia care - deborahO'connor) There is growing literature indicating that an approach to care that supports inclusion, recognition, trust, and the individual's personhood may delay or reduce several negative consequences that may develop with dementia. Tom Kitwood, one of the more prominent researchers to tackle the biomedical view, defined the personhood of the person with dementia as "the standing or status that is bestowed upon one human being, by others, in the context of relationship and social being" (Kitwood p.8). Instead of seeing the person with dementia by their disease, Kitwood challenged this by centring the personhood of the person with dementia by personalising care, paying attention to the individual's relationships, and maintaining decision-making by acknowledging the person's abilities.    While I describe the potential limitations of personhood in the later stages of this literature review, the concept of personhood has promoted necessary changes to dementia research and practice. For instance, the approach has brought forth the sharing of people's experiences with dementia that has redefined how we speak and involve people with dementia in research. Furthermore, the focus on lived experiences, has not only demonstrated the importance of individuality in care and when working with people living with dementia but has created a cultural and positive change towards those who have recently been diagnosed with dementia and began to address 'dementia worry' (Kessler et al., 2012).

An integral part of HCI work builds on personhood approaches where design and technology advancements have moved towards improving quality of life, supporting inclusion, evoking emotion, and engagement through creativity to help foster heightening subjective wellbeing, maintaining skills, and providing social engagement. While many creatively oriented technologies have relied on the person's ability to articulate past events or configure dementia as a series of problems, as researchers have moved toward the inclusion of the voices of people living with dementia, recent HCI research has similarly begun to question how to position people with dementia in the designing of technology appropriately. With this in mind, the following subsections review dementia-HCI literature that investigates the type of technologies, design, and participatory approaches researchers use in the domain.
% Add something here making it very clear what the nxt section is. Reviewing the literature -- highlight the current state of HCI literature + potential challenges / themes that the PhD will explore

\subsection{Technology design for and with people with dementia}
\label{BL:Tech}



