\chapter{Sharing a virtual world with people with dementia: A reflective account}
\label{NegotatingReseacherParticipantRelationships}

\section{Study one: Blending old with the new}
\label{StudyOne}
In November 2016, I started my final undergraduate year in the School of Computing at Newcastle University. Given my interests in  HCI and developing technology around dementia, the Head of School assigned my project to Dr Madeline Balaam and Dr Kellie Morrissey, who have similar interests in health and dementia. Given my family history with dementia, I was interested in exploring how technology could improve people's lives with dementia. My Grandpa was diagnosed with Alzheimer’s in his early 50’s, and my Grandma took care of him until he passed away when he was 67 (2001). I wanted to know more about the neurodegenerative condition and understand what my Grandpa and Grandma went through. 

At the time, virtual reality (VR) was gaining attention through the popularity of VR headsets and being picked up by the entertainment industry, particularly the gaming industry \citep{cipriani_understanding_2014}. When looking at the uses of VR for people with dementia, it was surprising to see a focus on neurological rehabilitation \citep{schultheis_application_2001,mendez2015virtual}. For instance, \citep{garcia2012discussion} proposed VR to offer brain-stimulating activities to reduce the progression of dementia. While this work is promising in their domains, at the time, prior work did not consider VR for people with dementia could function as an expressive and creative medium.

As such, by focusing on the growing body of work that has concentrated toward evoking emotion \citep{wallace_design-led_2013}, and creativity through technology with people living with dementia, this study aimed to consider how VR experiences for people with dementia might be sensitively designed to provide comfortable and enriching experiences. As I describe in the methodology, Sandra from Silverline Memories had also expressed interest in the design of Virtual reality on AppMovement where she describes the app as providing \textit{``images and scenes which could stimulate memory as well as providing comfort and reassurance to people with dementia or any memory loss''} (see figure \ref{fig:AppMovement-Sandra} for AppMovement quote). With Silverline Memories residing in the outskirts of Newcastle, my supervisors reached out to see if I could run a series of workshops at their dementia café as part of their afternoon tea sessions on Mondays. Dementia Cafés are places where people living with dementia, their families, and friends can come along and be part of a supportive environment that encourages opportunities for sharing experiences. These workshops had been organised to be flexible to co-exist alongside other organised activities within the dementia café. The aim was to get to know the members of Silverline Memories, and from getting to know one another, I could then curate a set of tailored VR experiences that would be interesting for the cafe.

\subsection{Workshop One: Getting to know the participants}
\label{StudyOne:W1}
For the first workshop, I wanted to find out what type of environments the members at Silverline Memories might want from a virtual reality experience. Working alongside Dr Kellie Morrissey, we set out to Silverline Memories dementia café, and I felt nervous. I felt so out of place. Apart from my experience with my Grandpa when I was a child, I had never really been around people living with dementia. Kellie and I arrived at the dementia café a little earlier than Sandra. Sandra came later than us with bags full of snacks and drinks for the members. As we approached the dementia café. We helped Sandra carry the bags into the the cafe (seen in fig x). On the left side of the room was a kitchen area for volunteers to hand out tea, coffee and biscuits throughout the session, which had an open plan for volunteers and members to help themselves freely. 

Silverline memories had never had the chance to make space their own, with the community room being shared across many different groups. Instead, you had a sense of' home' or' community' emerging through the interactions with the volunteers and members. As I had set up the room, I got my notebook out, VR headset, a recorder, and consent/information sheets. As the first couple entered the café, Sandra and the other volunteers came over to them with open arms – similar to everyone who walked in on that day. They caught up, got them a cuppa tea and biscuits, and sat down. Sandra introduced the couple, Philip and Kate, who seemed enthusiastic to talk to us. They asked about the research, and what we did. However, as I would find out later on, the initial few minutes of signing consent, reading information sheets, and explaining the research are uncomfortable for all those involved. At that moment, it went from an informal conversation into a formal study where two of us would be analysing and studying what was said. As we described the study, both were very happy to participate in the conversation about the types of VR experiences they would like to see. They were okay with quotes being used as long as they were anonymised.

With VR being relatively new to the members, I began by introducing a simple VR experience which consisted of being placed in a virtual apartment as participants tried on a Google Cardboard headset. The decision for an apartment VR experience was decided for its neutral nature; it did not give any low or high expectations for what to expect with VR technology. After participants tried the headset, we spoke about the type of places they would like to see through the VR headset. I used printouts of images to further these conversations, such as images of libraries, museums, forests, and beaches. During the first workshop, I spoke for an extended time with one couple, Thomas and Janet, where Janet was living with dementia. The couple told me about Janet's preferences for a VR environment that emphasised country music. From this, I decided to create a personalised VR experience based on her love for Shania Twain. I also set out to design and develop environments for the dementia café. The first was a beach environment, and the second was a park that took inspiration from a local park that participants had reminisced about in the workshop. Thirdly, as briefly mentioned above, I sought to design a bespoke Shania Twain concert hall experience for Thomas and Janet.

The workshop was my first experience working with people with dementia, but more importantly, the first time I would be seen as a' researcher'. Although I looked young and got sarcastic comments from the participants about my age, I did not know how to conduct myself in conversations with this new title. Each conversation felt like an interview, and I sensed a power imbalance. In some instances, power imbalance came from both sides, with members of Silverline Memories having been a part of the café for some time. Members did not have to talk to us or participate in the study, and if they did not, they had nothing to lose.

On the other hand, Kellie and I were the only ones walking around with the title of' researcher' who were here to interview and record participants. It was so unnatural to me, but why would it be natural? I do not start my conversations with friends or family with consent forms and placing an audio recorder on the table. Nevertheless, the methodologies I was told to follow focused on the importance of researcher-participant relationships. For instance, \cite{mckillop2004make} highlight the importance of building relationships with people with dementia to make participants feel more comfortable in the study. The authors provide several strategies to improve engagement in interviews, such as talking about the person's life and what they have done today; being empathetic and caring in the interviews; and being flexible in the conversation to fit the participant's needs.

