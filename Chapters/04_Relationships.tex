\chapter{Sharing a virtual world with people with dementia: A reflective account}
\label{NegotatingReseacherParticipantRelationships}

\section{Study one: Blending old with the new}
\label{StudyOne}
In November 2016, I started my final undergraduate year in the School of Computing at Newcastle University. Given my interests in  HCI and developing technology around dementia, the Head of School assigned my project to Dr Madeline Balaam and Dr Kellie Morrissey, who have similar interests in health and dementia. Given my family history with dementia, I was interested in exploring how technology could improve people's lives with dementia. My Grandpa was diagnosed with Alzheimer’s in his early 50’s, and my Grandma took care of him until he passed away when he was 67 (2001). I wanted to know more about the neurodegenerative condition and understand what my Grandpa and Grandma went through. 

At the time, virtual reality (VR) was gaining attention through the popularity of VR headsets and being picked up by the entertainment industry, particularly the gaming industry \citep{cipriani_understanding_2014}. When looking at the uses of VR for people with dementia, it was surprising to see a focus on neurological rehabilitation \citep{schultheis_application_2001,mendez2015virtual}. For instance, \citep{garcia2012discussion} proposed VR to offer brain-stimulating activities to reduce the progression of dementia. While this work is promising in their domains, at the time, prior work did not consider VR for people with dementia could function as an expressive and creative medium.

As such, by focusing on the growing body of work that has concentrated toward evoking emotion \citep{wallace_design-led_2013}, and creativity through technology with people living with dementia, this study aimed to consider how VR experiences for people with dementia might be sensitively designed to provide comfortable and enriching experiences. As I describe in the methodology, Sandra from Silverline Memories had also expressed interest in the design of Virtual reality on AppMovement where she describes the app as providing \textit{``images and scenes which could stimulate memory as well as providing comfort and reassurance to people with dementia or any memory loss''} (see figure \ref{fig:AppMovement-Sandra} for AppMovement quote). With Silverline Memories residing in the outskirts of Newcastle, my supervisors reached out to see if I could run a series of workshops at their dementia café as part of their afternoon tea sessions on Mondays. Dementia Cafés are places where people living with dementia, their families, and friends can come along and be part of a supportive environment that encourages opportunities for sharing experiences. These workshops had been organised to be flexible to co-exist alongside other organised activities within the dementia café. The aim was to get to know the members of Silverline Memories, and from getting to know one another, I could then curate a set of tailored VR experiences that would be interesting for the cafe.

