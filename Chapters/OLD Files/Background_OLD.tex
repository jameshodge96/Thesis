\chapter{Background Literature}
\label{BackgroundLit}

\section{Introduction}
\label{BL:Intro}
Within this literature review, I aim to describe prior work for understanding how people with dementia have been represented and involved within HCI work. This work has been divided into three themes to recognise the missing gaps within dementia and HCI work. Furthermore, as the four case studies are situated within different contexts, each case study chapter contains relevant related work to provide this chapter laying the frameworks for why we should begin to broaden the debate of dementia. 

Part one introduces the development of more relational and nuanced approaches to working with people with dementia from its once viewed biomedical model. Within this change, I describe the continued public misrepresentations of dementia that have caused a knock-on effect for those with dementia. From here, I develop an understanding of involving people with dementia in research and people with dementia advocating for change has shifted the tackling of representation of dementia to an individual level. I summarise this section highlighting that these challenges are highly social and politically complex and should be tackled in a more multidimensional fashion by recognising and including multiple narratives within dementia.

Part two moves from dementia work to consider similarities and lessons learned from working with other marginalised populations. I begin by mirroring similar challenges between dementia and marginalised populations through public representation and participation, highlighting the barriers to working within real-world settings within academia. From here, I describe a series of work within HCI devoted to 'empathy-building' to support designers, developers and researchers initial insight into working with participants from marginalised communities. I use this study to illustrate the importance of collaboration and engagement between those who are being designed for and those who are doing the designing, which further supports the need to broaden the discourse around dementia to understand barriers that we may encounter when opening up the discourse around the topic.

Finally, part three returns to the dementia work and highlights prior work that has raised importance to broaden the dementia discourse that supports the importance of unpacking the type of relationships and interactions people with dementia may have in day-to-day experiences, which in turn, may provide a more insightful and holistic understanding of dementia. I conclude this section with brief descriptions of the four case studies that aim to discuss the social and political complexities within broadening the conversation around dementia. This final section summarises the literature to highlight the missing gaps in the literature and introduce the succeeding chapters.

\section{Representing the experiences and views of dementia}
\label{BL:partOne}
Over the last 40 years, our understanding of dementia has evolved across several perspectives that have positioned and represented the person with dementia in several ways. What was once a bio-medical stance has gradually moved towards one that considers the socio-political and individual experiences of dementia \citep{bellass_broadening_2019}. That is not to say that the bio-medical perspective was necessarily inappropriate in representing dementia. Instead, the approach may overlook ways to adapt and ensure people with dementia may continue to live a meaningful life with their diagnosis. For example, the medicalisation of dementia has led to an awareness of strain on care partners towards policymakers, improvements in identifying and diagnosing dementia, and the development of medication to reduce symptoms of dementia \citep{doi:10.1080/13607863.2019.1693968}. However, bio-medicalisation has significant social ramifications on the representation of dementia by the public, medical professions, and individuals diagnosed or close to someone with dementia \citep{lyman_bringing_1989}.

For instance, bio-medicalisation has provided a series of labels  of someone with dementia. For instance, the terms 'patient'. 'demented' or 'sufferer' tends to signify that a diagnosis erodes our being-in-the-world \citep{hampson_dementia:_2016}, adversely affecting a sense of belonging and, therefore, a sense of belonging, a sense of self. Further, As dementia progresses, it often adds conflict between the person and their surroundings as they can become unfamiliar, and this can also cause difficulties with coexisting with others; this can happen in what was previously a familiar space such as a family home, a local community, or workplace \citep{langdon_making_2007,au_social_2009}. Challenges within previously familiar surroundings can cause issues for the person living with dementia who may feel less able to express and explore their identity \citep{john_killick_claire_craig_creativity_2012,kontos_embodied_2005}. However, to a certain degree, this is a self-fulfilling prophecy. As we live in a society that places great value on cognitive ability, many believe that people with dementia are poor at social contact, which then prohibits many from interacting with people with dementia at all, in turn \citep{killick_communication_2001}. 

Driven to tackle the hurdles emphasised by the bio-medical stance, Kitwood and others have followed person-centred approaches to dementia care that called attention to how we communicate with people living with dementia \citep{kitwood_towards_1992, dewing_personhood_2008}. Rather than questioning someone's cognitive abilities, person-centered approaches promote embodiment that brings attention to lived experiences of the body \citep{kontos_embodiment_2013}, and non-verbal communication is a key in ensuring the person living with dementia can experience their life to the fullest. In the succeeding subsections, we describe prior HCI and gerontology work in the field of dementia together by splitting the literature into the following subsections:
\begin{enumerate}
    \item Positioning of people with dementia in care, research and in conversations.
    \item The portrayal of the experience of dementia and who contributes to the narrative.
    \item The critical turn in dementia that brings forth political and relational importance to the area.
\end{enumerate}

\subsection{Involving people with dementia}
Individuals living with dementia typically experience problems with language, memory, movement, and other abilities \citep{bature_signs_2017}. As part of the changes that come with dementia, the individual's role within the family structure can change, as they become the care-receiver. Therefore, the impact of dementia can be troubling for both parties \citep{ryan_dementia_2009}. In particular, social activity can decrease, which entails several "knock-on" effects, such as a decline in emotional well-being, and increased social isolation and depression \citep{cohen-mansfield_utilization_2006,lee_just_2003,susan_krauss_whitbourne_adult_2011}. From early on, a diagnosis of dementia has often presented the person with dementia as a "patient", "demented", and "in need of care" [55]. Further, many health care professionals will prescribe social responsibilities to be placed upon the care partner and for the person with dementia to \textit{"give up work, give up story, and to go home and live for the time [you] have left"} \citep{doi:10.1177/1471301214548136}. The radical shift in representation of a person with a new diagnosis of dementia and burden of responsibility to care partners, family and friends presented early technological work that responded to the challenges of caring for people with dementia. 

For instance, the design of GPS tracking of residents with dementia in care homes has been heavily researched \citep{wan_design_2016}, focusing on inviting care partners and care-home staff as stakeholders that overlooked the stigmatising and restriction on personal autonomy that the invasion of privacy a GPS tracker may place upon someone with dementia. While tracking may offer the staff a reduction in workload, the lack of consideration of people with dementia's needs and desires deprived them of their personhood and changed their quality of life. Landau et al. describes initial findings of people with dementia only ranked third for more important decision-making processes \citep{doi:10.1080/13607861003713166}. Similarly, early work in care home architecture \citep{torrington2006has}, care processes \citep{rabins2006practical}, and creative therapy \citep{schmitt2006creative} prioritised those without dementia in the development processes with the potential of people with dementia invited in the evaluation phases of development. This has resulted in prior products and services failing to represent the desires and needs of people with dementia causing a lack of up-take and ownership of technology design \citep{higgins2013involving}. As described earlier, with the shift of person-centred approaches from Kitwood's personhood approach, the involvement of people with dementia in care and research practices gradually increased \citep{dewing_personhood_2008}. These practices brought attention to the individual nature of the person with dementia \citep{fazio_fundamentals_2018}; adapting embodied and non-verbal communication to include the person with dementia \citep{kontos_embodied_2005}; consent processes \citep{dewing_participatory_2007}; and acknowledgement that dementia is a complex, multifaceted experience that requires attuning our design and research approaches.  

These key considerations aimed to bring forth an individuals desires, history, independence, respect and caring treatment within their care \citep{fazio_fundamentals_2018}. These person-centred approaches initiated within nursing and social care, but gradually influenced other fields that seek to work with and for people with dementia. Early HCI and dementia work adapted person-centred approaches to ensure a focus on how we involve people with dementia in processes of technology design \citep{vines_configuring_2013,lindsay_empathy_2012, wallace_making_2013,suijkerbuijk_active_2019}. For instance, Wallace et al. Use a tailored approach that centred the importance of personhood by paying attention to a person's individual and unique experiences of dementia to design bespoke digital artefacts for the participants \citep{wallace_enabling_2012}. 

Similarly, Lindsay et al. describes the need for more interpretative data approaches for those at later stages of dementia, which in turn, may require longer-term projects and relationships to form throughout a study \citep{lindsay_empathy_2012}. The work by Kontos and Twigg explored ways to continue to experience the world and create meaning through embodied practices that they place at the forefront in their design \citep{twigg_dress_2013,kontos_expressions_2007}.  Given the overwhelming focus on cognitive deficits in dementia in design research to date \citep{lazar_critical_2017}, tasks which leverage creativity and expression can be valuable in allowing creative communication. Bauman and Murray (2014) further this notion by stating that we should consider the person as a whole, including the new experiences and skills which may come with what seem to be deficits:

\begin{quote}
\textit{    Being deaf is not automatically defined simply by loss but could also be defined by differences, and in some cases gain. \citep{bauman_l._&_murray_deaf_2014} 
}\end{quote}

Bauman and Murray address a social stigma of personal self-being lost in those with cognitive/communication deficits \citep{bauman_l._&_murray_deaf_2014}. Murray further highlights the perspective of personhood as a shift away from the unity of sense but toward social interactions of the person rather than their neurological changes. This early work stresses the required need to adapt co-design and participatory approaches to accommodate the differing communication needs of people with dementia, ensuring they are respected rather than infantilised \citep{salari_social_2001,john_killick_claire_craig_creativity_2012,vines_configuring_2013}.

\subsection{Representation of dementia}
\label{BL:RepresentationofDementia}
Until recently, the lack of involvement of people with dementia to share their experiences and influence research agendas has had a significant ramifications on the public view of dementia, the societal response and shaped the way people with dementia see themselves \citep{swaffer_dementia_2014}. Harmful language and stereotypes contribute to the stigmatisation of dementia \citep{reynolds2017stigma}. For instance, Molden and Maxfield describe the dementia worry and anxiety the public have of the possibility of developing dementia \citep{molden2017impact}. Although dementia is not a natural part of the ageing process, the ageist stereotypes of memory loss,' suffering', and cognitive loss contribute to self-stereotypes and public misrepresentation of dementia. 

O'Connor et al. Illustrate the tensions that arise from the 'get-go' of a diagnosis with dementia. The study highlights the hardships of living with and being diagnosed with dementia is highly entwined into societal complexities that family, friends, and the public treat the person differently. For instance, a participant - Angus, describes:

\begin{quote}
\textit{"My broker–I've been dealing with her for 25 years and she doesn't call me anymore. She calls my wife. For 22 years, she never even talked to my wife once. My wife answered the phone and she always asked for me, even on her investments. As soon as my wife told her–I'm the one who told her, I guess. And the next thing you know, she doesn't–and she doesn't do it on purpose. But she just–that's just the way it is (Angus)." (pg.48) \citep{o2018stigma}}
\end{quote}

Although this thesis does not explore or delve into detail about the specifics of a diagnosis of dementia, from my previous work in dementia and work that will be found in this thesis, several people with dementia who I've worked with have shared similar frustrations and challenges of friends, family, and the public altering or blinded by misrepresentations. Not only does this effect people with dementia on a relational and personal level, it can limit their involvement in everyday life. For instance. Pachana et al. highlight that \textit{"biases and stereotypes present in the general population}" may impact the infrastructures that surround and often impact underrepresented groups \citep{pachana_can_2014}. For instance, Ethical Review Boards (ERB) who are in place to ensure research follows standard ethical principles, protecting the participants, researchers, and research institutions could be prone to similar stereotypes popular in public knowledge. Individuals on ERBs that are unaware of such biases may focus on the aims of protection, as opposed to the approval of research that attends to such issues as agency and ensures meaningful participation \citep{participants_back_2002}.

Further complexity arises since ERB decisions vary even within the same country or region \citep{edwards_research_2004}. This is because the decisions and reasoning's are made at a university level and influenced by cultural and local norms and customs. Thus, the disciplinary changes in working with populations such as dementia are not necessarily matched at the level of those who make decisions about what research is and is not allowed when carrying out participatory work with participants who are considered 'vulnerable', this tension is a key focus that I attend to in \textbf{chapter five}.

To tackle the miss-representations of dementia, we have started to see researchers and media engage with people with dementia who are sharing their individual experiences \citep{peel2014living}. Sharing and narrating of people with dementia experiences has drawn attention to power, the practice of citizenship, inclusion, and relationships \citep{villar_giving_2019,bartlett_citizenship_2014}. Motivation to share their experiences of living with dementia seems to be twofold: writing allows reclamation of social identity through sharing their thoughts and feelings, and second, sharing their experiences helps not only family members but also the public to look past the diagnosis of dementia by demonstrating life continues to be rich and meaningful post-diagnosis \citep{schorch_designing_2016, ryan_dementia_2009}. Sharing lived experiences can also be seen in the recent proliferation of blogs, presentations, and personal books advocating \citep{bryden_challenging_2020, christine_bryden_dancing_2005,nolan_perceptions_2006} for changes in media and public portrayals of dementia, which might counteract dominant misconceptions about, and stereotypes of the condition. 

Ewick et al. describe how narratives can be subversive, and to a degree, the stories we tell \textit{"make visible and explicit connections between particular lives and social organisations" \citep{ewick_subversive_1995} (pg.222)}. Baldwin draws on these connections along with coining the concept as narrative citizenship \citep{baldwin_narrative_2008}. Still, for this to exist, it depends on the ability to tell a story through either:
\begin{quote}
   \textit{"a) being able to express oneself in a form that is recognisable as a narrative, even if one's linguistic abilities are limited… [and] b) having the opportunity to express oneself narratively" \citep{baldwin_narrative_2008} (pg.225). }
\end{quote}

 Furthermore, While this creates an opportunity for public engagement, the extent to which the 'public' are engaging with these narratives is underexamined, begging the question: how can these experiences be better positioned for societal change-making? Moreover, such advocacy work, despite its benefits, is often associated with strain \citep{suijkerbuijk_active_2019}, through the public questioning advocates credibility or the public only seeing the \textit{"social front"} where the \textit{"psycho-emotional consequences of taking action…went unseen in the public" } \citep{bartlett_citizenship_2014}. For example, Christine Bryden, a pioneering dementia advocate, provided her \textit{"brain scans in her PowerPoint presentations"} \citep{swaffer_but_2016} . 

With this in mind, embedding public engagement into design work with vulnerable groups such as people with dementia requires careful consideration. One suggestion argued by Dupuis et al. is to \textit{"actively involve them in decision-making to the fullest of their abilities, and support their involvement using whatever means necessary" \citep{dupuis_moving_2012} (p.431)}. However, in doing so, we must be aware of how we engage with such complex (and often stigmatising) topics sensitively while encouraging engagement by the public and others who influence and are influenced by those with dementia.  For instance, researchers in the field will have years of experience working with people with dementia, providing insight into the differing care models, ramifications of language and stereotypical perceptions of dementia, and the inaccurate imagery of dementia. Moreover, inviting a broader public to input on the topic and contribute to the representation of dementia may run the risk of having these stereotypes aired publicly or even perpetuated. However, broadening the debate on dementia is necessary to move beyond individualised experiences of dementia and explore a more critical perspective of dementia that recognises the social and political structures that enable (or disable) those with dementia to participate and be appreciated. In our following sub-section, I describe the critical turn in dementia that offers ongoing dialogue within the community of dementia alongside challenging the societal views of dementia \citep{lazar_critical_2017}.

\subsection{Critical turn in dementia}
\label{BL:CriticalTurn}
In the last ten years, dementia practice and research have begun to reflect on the essence of power relations, individuality, knowledge and the complexities of experiences between self and society \citep{bartlett_personhood_2007}. Within this work, studies have expressed people with dementia can actively engage within communities and have meaningful interactions if supported appropriately \citep{mockford2017development}. One aspect that has started to gain attention by policy directives, public and researchers, is dementia advocacy \citep{weetch_involvement_2020}. Dementia advocacy networks have presented a radical adoption of ensuring agendas, and dementia work are to be led and influenced by people with dementia to help foster empathy with and understand a person with dementia's experience.  For instance, DEEP - a UK network of people with dementia- redefines what people with dementia 'can do' by leading research, developing ethical governance, developing language guides, and providing a space for people with dementia to regain confidence and provide valuable contributions to future work \citep{diaries_deep_2020}. A dementia advocate - Wendy describes: 

\begin{quote}
\textit{"I may not remember what we said, I just know that we all share the same values, the same feeling that this project will change people perceptions of what people with dementia CAN achieve." \citep{davies2021dementia}(pg.17)}
    
\end{quote}

People with dementia being involved in advocacy networks are starting to challenge previous assumptions of ability and memory \citep{bartlett_citizenship_2014}, which resonates with Bartlett \& Connor work on Social citizenship that positions people with dementia more than a care recipient, and rather someone who has multiple social identities such as: activist, blogger, campaigner, friend, and public speaker. Recently, dementia activists have been recognised using online interactions, bespoke forums, and Twitter to raise awareness and further challenge the stigma surrounding dementia \citep{talbot_how_2020}. Dai and Moffatt's recent work on social sharing through community-based programs describes how future platforms involving people with dementia need to allow flexibility for \textit{"dynamic roles" }where individuals can flip between \textit{"storytellings, listeners, contributors" as the "activities [on the platform] evolve"} \citep{dai2020making} (pg.10). Likewise, Johnson et al. argue that these roles may need support from "\textit{various stakeholders in participating without burdening" (pg.127)  \citep{johnson_older_2019, johnson2020roles}} people with dementia. Similarly, \cite{daly2018shared} argues care partners and workers consistently overlook the shared decision-making despite the fact that it's a positive impact on the individual, which requires additional exploration in how we may facilitate interdependent relationships in shared spaces . 

However, ensuring that narratives are shared and recognised becomes challenging in later stages of dementia, where non-verbal and ambiguous interactions may be more present \citep{villar_giving_2019}. \cite{treadaway_sensor_2016} emphasise that recognition and appreciation are crucial even when leaning on tacit, creative activities to support non-verbal interactions. This critical turn is further supported in work by \cite{morrissey_value_2017}, taking an experience-centred approach that shifts the way we see people with dementia-related cognitive deficits as contributing to design choices. While prior work may focus on alleviating a person's cognitive deficit, the critical perspective widens our approach to inclusivity by celebrating what a person has to offer through more creative and engaging approaches, where those living with a broad spectrum of dementia-related changes can also participate \citep{lazar_critical_2017}. This recent work is harmonious with other academic work that calls for embodied discourse that recognises relational, interdependence, and reciprocation to design more inclusive approaches. For instance, \cite{kontos_citizenship_2016} work on embodiment and sexuality highlights the ethical importance of providing training and education for residents sexuality within a care home to provide sexual expression and support intimate relationships.

While this work is coming together to tackle stereotypes and stigma and raise awareness of current portrayals of dementia, the extent to which the 'public' are engaging with these dialogues is underexamined.  Knowledge transfer and dissemination have been raised several times in HCI work and workshops, particularly in sharing more relatable and engaging research insights. \cite{gray2020knowledge} designed a play \textit{Cracked: New Light on Dementia} - as an artistic exploration into theatrical and emotional exploration of how we think about dementia. Gray's analysis of the play, suggests that the audience \textit{"began to draw a trajectory towards a new kind of social engagement beyond the performance event itself"}. By providing a place to question and reflect on their assumptions, audience members were provided with a space for sharing and refining a more sensitive, nuanced narrative of dementia. 

Concerned by similar implications of communities outside of dementia not having the intricate understandings about dementia, HCI has provided several intergenerational works including inviting students to collaborate in co-design methods with care home residents to support meaningful exchange and interaction by developing life story work \citep{foley_student_2020} and storytelling projects \citep{hannan_zeitgeist_2019}. \cite{hendriks_challenges_2014} further supports the importance of designers and students building a relationship with the people we are designing for and with. The authors argue design decisions \textit{"emerge from the relationships designers build" (pg. 3)}. Likewise, in \cite{foley_student_2020} work on student engagement within dementia care, the authors describe that over time, students started to take "responsibility for the development of the relationships" while the person with dementia was \textit{"viewed as experts, with knowledge and stories to share beyond their role as a patient in care" (pg.9)}. 

However, while opportunities to work in more non-traditional settings such as care homes may be possible through university classes, these are often limited to a small, selected group of students or courses focusing on healthcare and psychology \citep{kinnunen_understanding_2018}, meaning that those who are primed to design sociotechnical systems of care (e.g., interaction designers, computer scientists, technologists more generally) lack a degree of familiarity with the sort of population they may end up building for. Finally, developing education such as these through intergenerational interactions has often relied on organisations or care homes to provide a community of older adults or people with dementia, both increasing the workload of already pressured social care organisations and limiting the potential of involving communities or individuals who are not part of those selected organisations. 

In reviewing the above, we see many strives from people with dementia, care partners and researchers have developed in response to tackling the representation and involvement of people with dementia. Prior work has highlighted the way people with dementia advocate for change and the appropriate alterations to methodologies to curate a broader narrative of what dementia entails. However, exploring how to support and educate those without lived experience must be considered to provide inclusive and meaningful engagement between communities. 

\subsection{Summary}
\label{Dementia:Summary}
As I have highlighted, dementia and HCI research is shifting away from person-centred approaches that focus on only the person's individuality with dementia and towards a more relational and interdependence model, providing a multi-complex perspective of dementia. Specifically, the shift highlights the importance of broadening the discussion on dementia in order to explore the socio-cultural relationship between people with dementia and others. From here, we can move towards a more inclusive relationship and understand how our neighbours and communities provide "interdependencies and reciprocities that underpin caring relationships" \citep{nolan2002towards}(pg. 203) - regardless of their diagnosis. 
 
The drive of advocates and support networks being run by and with people with dementia has begun to challenge the public view; fund dementia-led research; re-imagine what ethics is for academic institutions; and teach researchers and the public through blogs, keynotes, and books about individual experiences \citep{brown_dementia_2013}. While this collection of experiences provides a change in representation, it may continue to rectify a simplified view of dementia that may under-examine differences of dementia throughout livelihood of living with dementia at all ages and stages \citep{bartlett2010broadening}.

Furthermore, researchers must consider what it means to be an inclusive society, what it means to do inclusive research and to question the infrastructures that surround and often hold up our work. How do designers, developers, the public and researchers influence the everyday interactions for people with dementia, and how may these stakeholders take account for changing perceptions of dementia, and question how interactions between people with dementia and other communities may foster or erode meaningful interactions. To this end, our following section draws on work within other underrepresented groups to find inspiration in tackling the issues described above within dementia.

\section{Participation and representation in marginalised populations}
\label{MP}
Co-design and participatory design traditions have historically engaged with marginalised communities to highlight members' agendas and individual needs on rights, benefits, resources, and identity \citep{devito_social_2019,porter_filtered_2017,scheuerman_safe_2018}. This work has resulted in design practices that examine the acknowledgement of emotion in our research [4]; give careful attention to researcher-participant relationships \citep{clarke_digital_2013}, and create safe spaces to support the sharing of sensitive topics \citep{lazar_safe_2019,talhouk_refugees_2016}. Following this work, the acknowledgement and sensitivity required for participants depending on their desires, needs as a community, and the community's history that we are building on. Prior work has innovated many of our methodological approaches to fit our participants better: for instance, careful navigation of gatekeeping \citep{sanghera_methodological_2008}, awareness of biases and stereotypes \citep{marsden_stereotypes_2016}, and providing slower, longer-term projects to provide the time to build trust and a relationship between the researcher and participants \citep{foley_care_2019}. Beyond designing for individual needs and desires, working specifically with LGBTQIA*groups \citep{byron_apps_2019}, refugee and immigrant populations \citep{talhouk_syrian_2016}, those with mental illnesses \citep{birbeck_self_2017}, older people \citep{reuter_older_2019}, and reproductive rights advocates \citep{michie_her_2018} has contributed to a more inclusive design that invites \textit{"new ways of doing, making and inhabiting the situation of our world today"\citep{rosner2018critical} (pg.65)}. With this in mind, our following sub-sections describe the relations and learning's that echo similarities between how we design and support those with dementia and people in underrepresented groups. From here, I can illustrate broader challenges and opportunities for participation with people with dementia and understand how we may design tools that promote collaboration and inclusion between communities. 

\subsection{Barriers to participation}
\label{MP:barriers}
Recent work in HCI has highlighted the sensitivities implicated in working with marginalised populations. For instance, \cite{porter_filtered_2017} focus on social and relational interactions between disabled and non-disabled individuals when they encounter dating online. The study highlights the challenges and pressures dating apps place on social expectations where \textit{"disabilities, especially those with visible ones, are discouraged from allowing their body and any limitation thereof to fade into the background" (pg. 87:8)}. The authors describe the difficulties of a-typical filtering features designed by dating apps where it forced those with disabilities to 'passive filter' where they have to disclose their disability in their bio to filter out potential matches of those with particular attitudes towards disability as opposed to having a filter criteria based on those individuals attitudes. 

In this example, designing respectful and safe spaces to open discussion within dating may provide individuals with disabilities to negotiate disclosure of other stigmatised identities. Similarly, \cite{talhouk_syrian_2016}. described instances where participants tend to be wary of particular topics and express their views to avoid conflict with gatekeepers. The authors work within Refugee issues describes the careful tensions required for researchers working in the field where transparency and building trust into research are invaluable values that need drawn-out longer projects. HCI work has continued to work with sensitive settings that have put forth the need to working closely with participants, prioritising the understanding of the population over actionable processes \citep{group_patron_2019}, continued engagement with participants \citep{waycott_ethical_2015}, careful navigation of gatekeeping with vulnerable groups  \citep{sanghera_methodological_2008}, and involving the social ecologies of the participant to design for social cohesion \citep{talja2005isms}. Among the challenges of underrepresentation in research, this is further limited by the other structural complexities such as Ethical Review Boards (ERBs). 

Ethical principles applied by ERBs are often influenced by the philosophical basis of morality and established codes of conduct shaped by culture and society. These guidelines are put in place to ensure both the participant and research institute are informed and protected. However, as HCI approaches such as participatory design and qualitative work have adopted more 'in the wild' methods, the approaches have been regularly ethically questioned by ERBs. \cite{bell_censorship_2014} argues that many ERBs' approaches to ethics align more with biomedical and experimental scientific methods, which fail to reflect the multiple ways of generating knowledge that encompass the third wave of HCI \citep{bodker_when_2006,lazar_critical_2017}. Willig suggests that for qualitative researchers,\textit{"ethical issues arise from the very beginning of the research, they stay with us throughout our interactions with our research participants, and they continue to be relevant throughout the process of dissemination of the research findings"} \citep{carla_introducing_2013}, and call for an adaptable approach to ethical research design.

Ethical implications are added further by the addition of technology within HCI. For instance, \cite{meurer_designing_2018} discuss innovation problems and their impact on sustainability. The drive to create novel research not only puts pressure that forces technology solutions that may not be appropriate for the community, but research may be ill-judged on funding for continued support after the project ends. Researchers can still fall into technology-focused ethical difficulties even when the research may collaborate with large technology corporations. \cite{vines_our_2017} describe frustrations while using Google Glass in its beta stages where participants encounter many breaking bugs. These bugs ranged from poor battery life, to Google Glass updating itself while in use despite participants' wishes to the contrary. The frustrations \cite{meurer_designing_2018} discusses as funding ends are further echoed by \cite{vines_our_2017}  where technology supported by larger corporations no longer developed because the company no longer sees it as profitable. With these ethical complexities in mind, researchers in HCI should consider other ways to counteract the robustness and longevity of technology when the project ends. While this could be mitigated in large pools of funding where developers, designers, and companies are brought in, we must re-consider how research can be rewarding and create moments of real connection where perhaps the impact has less to do with the technology that is created, but instead the journey and involvement throughout the research process.

In recent years, several studies focusing on designing for and with marginalised groups have been strengthened by civic approaches; the work of  \cite{corbett_exploring_2018}, \cite{asad_tap_2017}, and \cite{olivier_digital_2015}. For instance, \cite{balaam_feedfinder_2015} wrote about FeedFinder - a location-based reviewing app that seeks to crowd-source breastfeeding-friendly spaces; suggested the app "offers a platform through which the qualities that categorise political conditions can become better known, and acted upon by public". Similarly, \cite{cazacu2020empowerment} highlight that organisational and co-design civic models provide processes for various actors such as institutions, citizens, and organisations to critique and provide spaces for public institutions to adapt their role in the ecosystem. Furthermore, \cite{reuter_older_2019} explored the use of resources a university is often rich in (e.g. technical competence, A/V equipment) to innovate within a radio programme for older people, leading them to encourage researchers to \textit{"consider participatory action research as a method of assistance in itself, complemented by technical innovation to facilitate processes in this space"}.The civic approach shares similarities to dementia, where participant engagement values the more relational approach than transactional interactions, treating participants as citizens and not solely consumers. 

This move to online-mediated change-making mirrors a trend in work on participatory platforms and digital civics that has begun examining and developing tools to promote change and support marginalised voices \citep{corbett_exploring_2018}. For instance, \cite{puussaar2018making}. developed a visual map-based querying tool to provide the public to interpret and understand open source datasets that are typically incomprehensible by non-professionals. Asad's work on similar civic platforms describes that there is an \textit{"obligation [for] designers and researchers to ensure our work aligns with existing efforts in our respective research communities"} \citep{asad_tap_2017} (pg. 6314). Civic and public engagement approaches to designing in socially complex contexts may promise to open up meaningful conversations to a broader public. However, to support collaboration and engagement between those who are being designed for and those who are doing the designing, we need further consideration into how we represent the experiences and views of marginalised communities for those who are in the decision and designing roles which we delve into in our following sub-section.

\subsection{Towards attending to difference through empathy-building}
\label{MP:empathyBuilding}
Much research in HCI has been devoted to developing approaches and other creativity support mechanisms, which have variously been used to support design research, creativity, and technological development and implementation. Within these approaches to support understanding the users we are designing with or for are a set of design approaches to provide 'empathy-building' \citep{thieme2014enabling}. \cite{ferri2017rethinking} present two general approaches to empathy in HCI: the technology or method to provide empathy as an outcome in its design and support a deeper understanding of the domain.

The first approach centers on products or experiences to foster empathy by the users. From our prior literature review of dementia work, breadth of work supports reciprocal relationships by adapting methods to provide the understanding and build empathy and provide platforms or approaches to empower people with dementia to share their experiences. For instance, \cite{gotsis2010smart} designed SMART-Games: a series of games to elicit empathy and related social skills for those with autism (SMART-Games). Emphasising empathy in games is not new, especially not in interactive media and education. Game design, education and HCI have explored the interactions with artefacts and games that can provide socially-aware persuasive storytelling, aspects of empathy and shape the attitudes towards a shared similarity or represent the difference between those who may have direct contact \citep{park2018systematic}.

In contrast, AI conversational agents have been recognised as an opportunity to provide empathetic relationships to empathise with the user. In times of global instability and precarity, AI has found its place in an attempt to help reduce care costs. As \cite{morris2018towards} Morris et al. describes, conversational agents may provide nuanced mental health applications that support the user in empathetic ways - and to some extent, match human capabilities. While the author's highlight the potential pitfalls this may provide, including whether the replacement of human-to-human relationships should be the case, there is a clear indication that the need to present empathy through products and experiences is one of interest to HCI.

The second approach emphasises designers, developers, and researchers' importance in gaining an empathetic understanding for 'knowing the user'. The interest in providing tools, approaches and methods to achieve a more empathetic understanding emerged to provide \textit{"more fertile ground for ergonomics recommendations and broader support for resolving critical issues"} \citep{suri2001next}. As McCarthy and Wright describe: 
\begin{quote}
\textit{"using empathy to understand the relationship between user and designer and to reflect on experience-centred design methods with a view to understanding how they mediate that relationship." \citep{wright2008empathy}(pg. 644)}    
\end{quote}

The drive to understand the relationship between the user and designer resonates with previous work described in this literature to provide empathy for the researcher and share the participants' experiences, desires, and needs. The growing interest in user-oriented research has guided the interest in designing empathy probes; experience-centred ethnographic approaches; telling participants narratives through documentaries and vignettes, and prototyping through simulation. For instance, the development of personas  for health-promoting services with vulnerable children \citep{warnestaal2014co} and VR simulations to aid in understanding what it may like to be \textit{'the other'}. While these activities are beneficial in prompting learning and informing user scenarios for the technology at hand \citep{vines_age-old_2015}, researchers must be wary that activities such as personas may reduce flexibility and creativity by attempting to fit the technology to a set of \textit{"caricatures"} \citep{redstrom_towards_2006}, rather than exploring the ambiguous nature of how people may use the technology. 

Recent HCI work has highlighted that even though empathy is necessary for sensitive contexts, it is not enough to rely on the researchers' empathy. \cite{spiel2017empathy} argues that the experiences of autistic children may require multiple viewpoints to provide more meaningful involvement and experiences as opposed to relying on the empathy of the researcher. Similarly, \cite{foley_student_2020} dementia work mirror this by describing the necessity of growing empathy over time and reflecting on purposeful everyday activities to provide students with the mutually engaging relationship between themselves and the care-home resident. With this in mind, embedding empathy-building into design work with vulnerable groups requires careful consideration. One challenge lies in how we engage with such complex topics sensitively while encouraging engagement from designers, developers and researchers. One suggestion by \cite{bennett_promise_2019} may be to see empathy-building as 'being with' instead of 'being like'. In this instance, we may look beyond simplified roles of someone with 'dementia' or 'designer', and in turn provide spaces for sharing and collaborative learning between different communities, which may then be realised in the products, services, and systems we co-create.

\subsection{Summary}
\label{MP:summary}
This thread of literature has sought to explore the types of engagements and barriers other researchers have encountered in working with marginalised populations. We can see from this work that they are similarities between the challenges and opportunities faced within the marginalised population and engagement with dementia. \cite{hope_hackathons_2019} emphasises the importance of restructuring public events to accustom marginalised populations, to provide ways to support education to those without lived experience. However, the challenges lie in curating these platforms or tools to promote change. It requires a diverse team of experts who likely have designer and developer skillsets and experience the individuals they are designing for. One area of work that is growing and of particular interest to this thesis is the development of platforms and services to promote change and support marginalised voices. While Digital Civics describes work that places great importance on providing marginalised groups the ability and skillset to participate in political and social change, little work has been done to raise our awareness of how designers, researchers and developers may influence and/or hinder our respective research communities. 

To provide insight into broadening the discourse and begin dialogical conversations centred on dementia, there is a challenge for balancing the voices and experiences of people with dementia and their relationships. The final section describes the social citizenship model that offers attention to the more relational complexities within dementia to provide insight into prior work that captures a more sociopolitical perspective. Furthermore, the section concludes with an overview of the four case studies within the thesis that tackle the representation of people with dementia within HCI.

\section{Broadening the dementia discourse}
\label{BL:Discourse}
Over the past 30 years, our understanding of dementia has gradually moved beyond what was once a biomedical view by the public. Since then, researchers, and people with dementia have shared and published personal experiences of living with dementia that has brought forward improvements to care and representation of dementia \citep{smebye_influence_2013}. In recent years, people with dementia have formed advocacy groups to challenge public perceptions and redefine what it means to have dementia - a lens of hope that presents a view of dementia where people can continue to experience the world around them and contribute socially and political contexts \citep{bartlett2010broadening}. As we are beginning to move into a new era of dementia and HCI work, we must broaden the dementia discourse that moves beyond the individualised narrative that has been promoted and popularised over the last decade. 

As we consider how we may broaden the debate but continue to lead from a person-centred approach, we must consider other lenses in research, such as the citizenship lens that Bartlett and O'Connor have recently popularised. Social citizenship has emerged as a helpful way that is defined as:
\textit{"a relationship, practice or status, in which a person with dementia is entitled to experience freedom from discrimination, and to have opportunities to grow and participate in life to the fullest extent possible. It involves justice, recognition of social positions and the upholding of personhood, rights and a fluid degree of responsibility for shaping events at a personal and societal level' \citep{bartlett2010broadening}(p. 37)."}

In the simplest terms, citizenship is a 'status bestowed on those who are full members of a community. All who possess the status are equal with respect to the rights and duties with which the status bestows' \citep{marshall_class_1964}. Emphasis is placed on the 'membership' of being part of a community where others' bestow' citizenship upon you. As Bartlett et al. argue, the idea of 'status bestowed' echoes Kitwood's definition of personhood, where others around the person living with dementia have the responsibility to maintain and sustain the individual's personhood \citep{bartlett_personhood_2007}. A significant difference between the two lenses is citizenship acknowledges the power dynamics that are in place between individuals and their relationships and, in particular, how relationships are relational and flexible. Let's consider the Mental Capacity Act (2005). It attempts to establish and maintain people's citizenship as we are required that every effort is made to involve the individual in their decision-making processes \citep{oyebode_mental_2005}. Even when this may not be possible, a carer must decide based on their 'best interest' that must consider the individual's views, relationships, change of the person's condition, and finally, how the decision will avoid restricting the person's life. In this particular example, the Mental Capacity Act emphasises the identity, and relationships that 'link citizenship with the possibilities of participating on an equal basis' \citep{ebersold_affiliating_2007}.

But this view of citizenship is more concerned with the 'civic virtues', that is, the right to vote, attending book groups, volunteering but does very little for the practice of citizenship. For example, people living with dementia have the right to vote, but the difficulties in voting can undermine that right \citep{lister_citizenship_2017}. As dementia has a vast variety of different neurogenerative conditions, the ability to travel to vote in person or having to depend on a carer to assist in posting your vote causes tensions on active citizenship or independence. While it is integral to recognise rights, in recent years, we have seen a shift towards the awareness of how individuals influence their community that has less to do with a status that is 'bestowed' upon them. More to do with actively involving themselves in the practice of citizenship \citep{lister_citizenship_2017}. This shift on incorporating citizenship as a practice can be seen in disability studies that echo similar challenges that people living with dementia have encountered. The disability rights movement raised awareness of how people with physical disabilities would be viewed in a 'medical' or 'sympathetic' model. Instead of these lenses, the movement expressed their right to be entitled to the same life as everyone else and not to be singled out \citep{bartlett_personhood_2007}. 

The difficulty in taking a citizenship lens into dementia research is that it undermines the notion of individuality that personhood has been praised for. Dementia should never be considered as a 'group' per se. Still, instead we should take reference to \cite{bauman_l._&_murray_deaf_2014} stating that we should consider the person as a whole, including the new experiences and skills which may come with what seem to be deficits. Kontos argues extending the social citizenship model by drawing on embodied selfhood and relationships - called model of relational citizenship. In this way, 'flourishing' of someone with dementia is "supported in and through the creation of enabling environments and relational practices – or corporeal-ethical spaces – that support embodied forms of communication and meaningful engagement" \citep{macpherson2011guiding} (Macpherson 2016). To acknowledge this way of thinking, we must consider the structures that influence day-to-day experiences and how knowledge and dementia narratives are perceived within the public. Therefore, broadening dementia discourse to unpack the social and political relationships for people with dementia will offer critical investigation into the support and representation of people with dementia in and outside the space of HCI.

\subsection{The shift towards a critical dementia perspective}
\label{BL:shift}
A critical approach to the narratives of dementia and HCI work has implications for how we view dementia, how influential other individuals are for the day-to-day experiences of people with dementia, and ways people with dementia may contribute and engage socially within society. This thesis presents four case studies to discuss the social and political structures that enable (or rather disable) those with dementia to participate and be recognised for their meaningful contributions. The four studies are not intended to be exhaustive but rather to move towards a multidimensional understanding of dementia and question the adapting roles and situations people with dementia may have within their relationships. While I describe prior literature focused on the four case studies in more detail within each of its chapters, the four areas of interest inspired by my literature review are the following:

\subsubsection{Attending to mutual collaborative relationships}
\label{BL:chapter4Overview}
Within the person-centred approach, there is a great emphasis on the independence and individuality of the person with dementia. The autonomy of an individual's decisions is an integral aspect of citizenship - one that has been examined within disability research and within dementia \citep{meissner_-it-yourself_2017, samsi_everyday_2013}. However, as dementia is a progressive condition, the change and decline poses challenges to the person with dementia's decision-making processes. While it is not uncommon for a partner to take it upon themselves to assist or take over the individual's roles within the family, several important decisions must be carefully considered, such as care planning, care-home placement and eventually - end of life \citep{fetherstonhaugh_decision-making_2017}. 

\cite{fetherstonhaugh2013being} highlight the invaluable importance for people with dementia to be involved in crucial decision-making choices that require the care-partner to learn how to enable those decision-making processes best. Attending towards more interdependent and supportive strategies requires careful consideration towards knowing the cared individual and continued alteration between subtle support and taking over. Within chapter four, this exploration into people with dementia and their families emphasises the importance of research that attends to the person with dementia's interests and desires and towards the family and friends \citep{keyes2019living}. Through this work, I  explore the day-day interactions and understand how technology can provide more nuanced and shared experiences that provide a more enabling role of shared decision-making. 

\subsubsection{Navigating barriers for involvement}
\label{BL:chapter5Overview}
With dementia's varying cognitive changes, much of the cognitive and social consequences of living with dementia can be framed as an ethical concern for the person at the heart of the condition and their family, making it a complex space for research and care practices. Research has called attention to the importance of ongoing consent \citep{dewing_participatory_2007}, longer-term projects, contested use of lies and deception in care \citep{lorey_fake_2019}, and attuning the need for embodied, non-verbal communication to ensure the person with dementia is respected and engaged within their care \citep{morrissey_value_2017,kontos_embodied_2005}. 

The ethical consideration needed to include the voices of people living with dementia in HCI research has resulted in a strong relational basis for design practice \citep{wallace_enabling_2012,houben_foregrounding_2019}. The established state-of the art based within this work has moved away from the biomedical deficit model of dementia, resulting in several underlying person-based values in design practice, many of which stem from the work of \cite{kitwood_towards_1992} and \cite{brooker_what_2003}. These practices include treating the person living with dementia as an individual in context; including the person living with dementia in research processes that aim to improve their quality of life; and acknowledging that dementia is a complex experience that often also includes social complexity, ageing and multi-morbidities, which require attuning to in design and research responses. These practices and design decisions offer particular ethical stances that appear essential to the success of HCI projects in this context and ensure the researcher-participant relationship is navigated with mutual respect and care \citep{foley_care_2019}. Making these ethical decision-making processes more visible within our empirical work has the potential to critically inform the current institutional and relational ethical framing in which we currently work \citep{oyebode_mental_2005}, and make more apparent considerations needed to ensure meaningful and engaged research with ‘vulnerable’ user groups is central to the design of technologies and systems.

\subsubsection{Change in public perception}
\label{BL:Chapter6Overview}
Public design events such as hackathons \citep{olesen_what_2021}, design sprints and workshops, involving as they do interdisciplinary teams interested in innovation, have been said to ’offer new opportunities and challenges for cooperative work by affording explicit, predictable, time-bounded spaces for interdependent work and access to new audiences of collaborators’ \citep{filippova_hacking_2017}. In recent years, various domains have adopted hackathons to no longer prioritise software or hardware skillsets by attendees. For instance, ‘civic hackathons’ \citep{johnson_civic_2014} have aimed to improve citizen-government relationships through transparency and open data and events focused on 'social good' \citep{ferrario_software_2014}. Within HCI, hackathon research has demonstrated useful cases for participation, learning, building, and connecting people in communities of practice \citep{falk_olesen_10_2020,hou_hacking_2017,trainer_how_2016}. However, such events pose challenges of longevity \citep{birbeck_self_2017}, compensation \citep{endrissat_hackathons_2018}, accessibility \citep{hope_hackathons_2019}, and representation of the area for which attendees are designing \citep{toombs_hackerspace_2017}. As hackathons have continued to be explored in HCI, researchers have re-structured and tailored the format to tackle the challenges described above. For instance, \cite{hope_hackathons_2019} leverage feminist and intersectional lenses to suggest pathways to building more inclusive and accessible events. 

With this in mind, hackathons that are embedded in sensitive settings must require careful consideration. 
One challenge lies in how we engage such sensitive topics while encouraging public engagement, which facilitates opportunities for collaborative learning and awareness around the topic of interest. While it takes an extensive period for researchers to become aware of the challenges and opportunities within the populations they are working alongside, hackathon formats expect the same sensitivities to be presented within a short amount of time – usually a weekend. Therefore, it is not a surprise that prior work has indicated that some design outputs may be unsuitable or feed into the stigmatising ideas of the group or topic at the centre of the design event \citep{toros_co-creation_2020}. Prior work has considered ways to sensitise attendees on the event's topic through presentations, workshops, and inspiration packs \citep{birbeck_self_2017} to upskill participants who may hold outdated or stereotypical attitudes towards the topic out of a lack of experience. Given the potential challenges of using such public design events \citep{toombs_proper_2015,kienzler_learning_2017}, it is particularly timely to begin to unpack the shifting representations and attitudes that influence attendees and their design outcomes to raise future considerations for designing hackathons within a sensitive context.

\subsubsection{Promoting collaborative learning}
\label{BL:chapter7Overview}
Much work in design and HCI has been devoted to the development of toolkits for designers and developers of interactive technologies. These toolkits have been used to support design inspiration, ideation and the implementation of technology \citep{broderick2020theory,jarusriboonchai2018thinking,ledo2018evaluation}. They come in many different forms, from card games \citep{peters2020toolkits,logler2018metaphor,alshehri2020scenario}, open-ended DIY resources  \citep{meissner2018schnittmuster}, to interactive games to evoke creativity and learning \citep{ellis2021tapeblocks}, and the simplification of complex frameworks or algorithms to aid practitioner and public understanding\citep{srinivasan2012cultural}. Within this body of work, researchers have highlighted how the collaboration of expert curators and the application of adaptable co-design approaches are often necessary to design an effective toolkit within certain domains. For instance, \cite{krafft2021action} describes a yearlong process engaging with several 2 partnering organisations, a team with diverse expertise, and a participatory action research approach for designing an Algorithmic Equity Toolkit. This toolkit was then comprised of definitions of AI, flowcharts, worksheets and probing questions to be posed to government agencies and policy-makers \citep{katell2020toward}. Critically, Lee and Singh argue some toolkits may be overbearing in terms of their information load and intended use, and raise a challenge for toolkits to be made “plug and play”, within the designer or developer’s workflow \citep{lee2021landscape}. Given these complexities around adaptability and expert curation of resources, understanding the role of co-curating in toolkits may allow emerging research in this area to interrogate the validity of toolkit components more critically, as well as their potential to offer impactful insights in the design process. While shared collaboration and curation has recently been explored in the development of a COVID-19 toolkit \citep{braybrooke2020together}, it remains to be seen how these types of activities represented in toolkits and creativity support when working with sensitive topics, where there may be issues around ethics and practicalities of participation.

This raises the question: How can toolkits and other creativity support tools foster dialogical engagement between people with dementia and designers and developers? Moreover, despite the benefits of people with dementia sharing their lived experiences and pushing for societal change, \cite{johnson_older_2019}. draw attention to the strain and potential “burdening” that this can cause, as people with dementia already often feel as though they need to support others living with the condition and actively challenge misconceptions and stereotypes that the public may hold. This raises interesting challenges about how best to ethically engage this community in design processes without overburdening their participation.

\section{Literature review summary}
\label{BL:summary}

This literature review has focused on research surrounding how we represent the experiences and views of marginalised communities - particularly that of dementia. The literature review started with a discussion on the ongoing shift from a bio-medical approach to a more person-centred approach that has been adapted and adopted by the HCI community. Through this discussion, I have highlighted the breadth of work done by researchers and people with dementia to tackle stereotypes and misrepresentations of a diagnosis of dementia. However, I illustrate that the current dementia narrative about individuality may be doing more harm than good. It limits our understanding of the more nuanced social and political complexities of living with dementia. However, I conclude this literature thread raises the question of how we may broaden the conversation around dementia to provide a more in-depth understanding of the relationships between people with dementia and others.

To provide HCI work that illustrates the broader challenges and opportunities for participation, I introduced literature that looks at participation and representation in marginalised populations. In this section, I describe the ethical challenges faced when infrastructures such as ERB's collide with the researchers and studies intentions to work closely with those in underrepresented groups. From here, I presented a series of examples of work that has been strengthened by a civic approach such as Digital Civics. This set of work highlighted the more relational approach providing tools and services to promote change and support marginalised voices. However, this exploration highlights the challenges faced where the tools and services require a highly diverse team of experts consisting of designers, developers and individuals with experience of working with the people they are designing for. In response, I reviewed HCI work that supports understanding the users we are designing with or for through design approaches that provide 'empathy building'. Although empathy-building can be helpful, One challenge lies in how we sensitively engage with such complex topics while encouraging engagement from designers, developers, and researchers.


Furthermore, the final section returned to the dementia literature to describe the more recent models of viewing dementia phrased as social/relational citizenship. Through this lens of seeing dementia, researchers encourage others to broaden the dementia discourse to unpack the social and political relationships for people with dementia will offer critical investigation into the support and representation of people with dementia in and outside the space of HCI. 

The following chapter of this thesis contains *insert paragraph here once methodology finished*
