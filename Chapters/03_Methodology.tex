\chapter{Methodology}
\label{Methodology}

\section{Introduction}
\label{Method:Intro}

In the previous chapter, I have described the evolution of dementia-HCI research in dementia that once questioned the biomedical view of dementia towards a political model of dementia that encompasses individuality, relationships, and supporting decision-making. Additionally, I discussed the relevant ways researchers engage with lived experiences of dementia by constructing safe spaces that lean on creativity and longer-term engagements to support relationships between the researcher and the person with dementia. The literature then moved outside the HCI space to examine the type of ethical dilemmas, public perception and ways to adapt methodological approaches to be more inclusive. Finally, I conclude the literature by providing a set of avenues of investigation that might broaden our understanding of designing and developing technology for and with people with dementia. Consequently, these avenues are examined in the PhD.

In this chapter, I describe the methodological approach taken in the thesis that stretches across the data chapters. I begin the chapter by introducing the epistemological approaches adopted in this thesis. Following, I describe the influence Participatory Design and co-design approaches have on the methodology \citep{duarte2018participatory}. In this section, I describe these design approaches relevance in HCI. Following, I unpack participatory methods in dementia and HCI that emphasise the approach's opportunities and pitfalls. I then discuss the ethical complexities of the thesis; an overview of the data collection of the four data chapters; overview of recruitment and location; and the qualitative data analysis method, thematic analysis and reflexive practice. This chapter closes by summarising the methodology for the thesis and returning to the research questions.  

\section{The social construct of dementia}
\label{social construct}
As part of the changes that come with dementia, significant social ramifications can also cause people with dementia's experience of the world to worsen \citep{hampson_dementia:_2016}. Challenges within previously familiar surroundings can cause issues for the person living with dementia, who may feel less able to express and explore their identity \citep{john_killick_claire_craig_creativity_2012}. As we live in a society that places a high value on cognitive ability, a diagnosis of dementia can put significant strain on meaningful interactions, relationships, and activities. Authors have argued that when a person has dementia, their cognitive dysfunction erodes our being-in-the-world \citep{hampson_dementia:_2016}, adversely affecting a sense of belonging and, therefore, a sense of self.

Alternatively, \cite{gallagher_merleau-pontys_2010} recognised that individuals with a decline in cognitive abilities can continue to experience the world and create meaning. This view is further explored in the context of dementia by \cite{kontos_embodied_2005} and \cite{twigg_dress_2013}. This shift puts the body and embodied practices at the forefront of design. Given the overwhelming focus on cognitive deficits in dementia in design research to date \citep{lazar_critical_2017}, tasks which leverage creativity and expression can be valuable in allowing creative communication. \cite{bauman_l._&_murray_deaf_2014} further this notion by stating that we should consider the person as a whole, including the new experiences and skills which may come with what seem to be deficits:
\begin{quote}
    \textit{Being deaf is not automatically defined simply by loss but could also be defined by differences, and in some cases gain.}
\end{quote}

\cite{bauman_l._&_murray_deaf_2014} address a social stigma of personal self-being lost among those who have cognitive/communication deficits. Murray further highlights the perspective of personhood as a shift away from the unity of sense, but toward social interactions of the person rather than their neurological changes.

As described in the literature, several social constructs emerged to tackle the medical model to emphasise models of care, individuality, and citizenship. In \cite{bosco2019social} systematic review on the social construction of dementia, the authors report that the meaning of dementia is deeply rooted in social-cultural contexts, which ultimately has potential effects of stigma and misrepresentation of dementia. Similarly, the stories researchers write and analyse can also be bound to the researcher's unknown biases, experiences and meaning-making process resulting in narratives being \textit{"rejected, reformulated, and reclaimed"} \citep[pg.850]{mcgovern2014forgotten}. Applying a social constructionist approach to this research places focus on the participant being the expert on the topic rather than the researcher \citep{padgett2016qualitative}.

Based on my understanding, I took a social constructionist approach to this research, in which the creation of knowledge is a collaborative process. Therefore, the way I structured the study and interpreted the data is significantly influenced by my relationship with the participants and my understanding of dementia \citep{surr2006preservation}. Social constructionism argues that the self is formed through language and interactions of varying kinds that are equally important in defining the person's individuality \citep{sarup1996identity}. Viewing the data this way recognises the nonverbal and actions of people with dementia with value and intention. Further, given that the thesis focuses on the perspectives of diverse stakeholders involved in the design process, constructionism recognises that 'multiple knowledges' exist together that shape and transform reality through social interactions and language \citep{mckeown2015you}. With this in mind, the following section examines the value of lived experiences and how these shared stories build a more positive construction of dementia.

\section{Value in the lived experience}
\label{Method:experience}
For the past 20 years, it has been accepted that dementia is not solely reliant on someone's cognitive abilities but rather made up of a set of bio-social-psychological factors that calls attention to centre the individual with dementia in research, practice and their care \citep{dewing_personhood_2008}. The shift in recognising the person with dementia's individuality emphasised the value in engaging and understanding the lived experiences and stories of people with dementia and their care partners. Recently, \cite{bartlett_personhood_2007} have moved beyond solely individual values through a citizenship lens that recognises the potential power relationships that will likely stem from a diagnosis of dementia. The authors argue that a lens that considers not only the power dynamics but also the relationships to the person and the unique nature of the individual are all connections we must consider when reflecting on the dementia context. Moving towards a citizenship model has empowered and promoted people with dementia to share their experiences to make themselves more visible to impact policymaking, practice, and research. The sharing of experiences has an impact in two ways. First, advocating and sharing experiences foster purpose and impact in understanding dementia on a more meaningful and inclusive level; and two, the stories promote awareness and improve public perception.

\cite{ewick_subversive_1995} describe how lived experiences can be subversive, and to a degree, the stories we tell \textit{"make visible and explicit connections between particular lives and social organisations"}. These stories that we tell of our respective worlds, open up new ways of being and may drive the lens that we view dementia, towards one that puts forth narrative, personhood, and citizenship where no lens overshadows another but instead work together in harmony \citep{dupuis_re-claiming_2016}. Baldwin draws on these connections along with coining the concept as narrative citizenship \citep{baldwin_narrative_2008}. Still, for this to exist, it depends on the ability to tell a story through either:
\begin{quote}
\textit{ "a) being able to express oneself in a form that is recognisable as a narrative, even if one's linguistic abilities are limited. 
b) having the opportunity to express oneself narratively" \citep{baldwin_narrative_2008}.  }
\end{quote}

The lack of involvement of people with dementia's experiences in research has primarily come from the expectation that people with dementia do not communicate to 'social norms' - promoting stereotyped narratives that have further excluded people with dementia. Returning to Baldwin, the researcher highlights the use of arts and creativity to provide alternatives to verbal communication that extends the importance of finding unique ways to offer people with dementia the opportunity to share their experiences. Through humour, dancing, acting, music, movement, and fashion, people living with dementia can evoke narratives and identity that can be particularly effective for those who may have their narratives' dissolve' as the condition develops \cite{john_killick_claire_craig_creativity_2012}. 

As the narrative of dementia continues to grow and change through the sharing of lived experiences, we must continue to explore ways to represent the voices of those who are continued to be underrepresented through creative ways to involve these individuals in the conversation and amplify their voices. \cite{swarbrick2015quest} argues research approaches should be more collaborative where people with dementia are recognised for their contribution. The following sections describe the growing interest in participatory design methods and their adoption in dementia research to enable participants to participate in the study actively. 

\section{Participatory Design}
\label{Method:PD}
Participatory design (PD) has been influential in various social research fields, including HCI \citep{bannon2018introduction}. As the name implies, PD is about the \textit{design} of systems, products, or knowledge through \textit{participatory} methods to understand how users may interact or use an artefact or practice. These methods draw from ethnographic observations, interviews, focus groups, and qualitative content analysis. \cite{carroll2007participatory} articulate the importance of user participation from the following:

\begin{quote}
"The 'users' – that is, the people who stand to have their activity and experience transformed – ought to have a direct say and a meaningful role in how that comes to pass at the very least because they know a lot about what is precious and what is annoying in their current activity and experience, but equally because they are morally entitled to have a say in anything that might change everything." \citep{carroll2007participatory}    
\end{quote}

Through these participatory approaches, PD aims to create a collaborative space between developers and the population group that was often separated from the design stages due to different experience levels \citep{duarte2018participatory}. Subsequently, PD has a significant political history where early PD work in the 70's \textit{"sought to rebalance power and agency among managers and workers" } \citep{bannon2018introduction}. Although PD has radically altered from what it was in the 70's, PD is still a popular approach to tackle communication barriers between different expertise levels. 

Within the general structure of PD, \cite{halskov2015diversity} argue they are five fundamental aspects of PD:
\begin{enumerate}
\item politics - \textit{"people who are affected by a decision should have an opportunity to influence it"}
\item People - \textit{"People play critical roles in design by being experts in their own lives"}
\item Context - \textit{"The use situation is the fundamental starting point for the design process"}
\item Methods - \textit{"Methods are means for users to gain influence in design processes"}
\item Product - \textit{"The goal of participation is to design alternatives, improving quality of life"}
\end{enumerate}

Halskov \& Hansen review highlight that HCI and other fields are diversifying and rethinking what participation may be within their domains, resulting in a highly diverse set of studies that reconfigure methods to fit the needs of participants to provide participation - particularly those who are considered marginalised, such as children, older adults and those who have varying cognitive deficits. For instance, \cite{spiel2018micro} conducted a series of PD studies with marginalised children that describe a set of ethical challenges of conducting PD within the space and the necessity to tailor participation for the children. The authors present detailed insights into tailoring PD processes to negotiate \textit{"what the children can do and the desires they have"}, design safe and attuned spaces \textit{"for the participation of children on their own terms"} and "\textit{make ethical judgments" }that are attuned to kindness and learning. Furthermore, \cite{vines_configuring_2013} describe several issues that PD needs to address. These challenges centre on the shared control and agency between researchers and participants. For instance, the author's highlight the need for transparency of participant and researcher's roles; how researchers present and analyse participants contributions; and further examine how participants can take agency of the design process and reinvent methods to suit their needs.

As described above, while PD literature describes the flexibility in adapting to many different populations, communities that may be marginalised often come with complex, unforeseen tensions that require sensitive and personal consideration to provide collaborative engagement between the researcher and the marginalised community \citep{harrington_deconstructing_2019}. With this in mind, the following section provides insight into the adaptation of PD approaches to fit the needs of people with dementia.

\section{Reconstructing participatory design methods for people with dementia}
\label{method:DementiaPD}
By tradition, early research in dementia has typically use focus groups, interviews, and involving stakeholders who are 'experts' in the field such as care-partners to make design decisions on behalf of people with dementia \citep{branco_personalised_2017}. This take on dementia research has often seen dementia as a cognitive problem, and in turn, constructed an assumption that people with dementia cannot participate due to verbal and communication issues. As we live in a society that places a high value on cognitive ability, a diagnosis of dementia can put significant strain on meaningful interactions, relationships, and activities. Authors have argued that when a person has dementia, their cognitive dysfunction erodes our being-in-the-world \citep{hampson_dementia:_2016}, adversely affecting a sense of belonging and, therefore, a sense of self.

However, as our understanding of dementia has changed over the years, so have our methods of involving and presenting the lived experiences of people with dementia. \cite{john_killick_claire_craig_creativity_2012} work in Creativity and Communication in Persons with Dementia, describe the importance of expression through the arts: 

\begin{quote}
"Creativity is an expression of who we are, and when the arts form the vehicle or the means of channeling this creativity, the end result can embody something of the artist and their facets of personality" \cite{john_killick_claire_craig_creativity_2012} pg.17
\end{quote}

The authors highlight alternative ways that people with dementia can participate and express themselves, such as - laughing, music, writing, and painting. \cite{ryan_dementia_2009} draw on creative methods for communication and expression through writing as a way to reclaim some sense of social identity, structure, and clarity on distinct thoughts and feelings. Furthermore, writing projects a sense of self to loved ones, helping family members see past a relative's dementia and drawing attention to the creative potential for the person living with dementia. 

Likewise, some HCI research has expanded to explore how new technologies can attune to the creative wishes of people living with dementia. \cite{lazar_critical_2017} takes a critical dementia perspective to focus on how arts-based activities enable researchers to learn and draw from the experiences of people living with dementia. The author's work into understanding the way art therapists configure the space for engagement noticed ways to encourage those living with dementia to express themselves by simply being in the moment. Taking part was noticed through \textit{"subtle shifts in gaze, facial expressions, and verbalisation"} and highlights the importance of tailoring the environment such as the brush, canvas and colours to fit the individual. The individuality in approach, suggests an enable for \textit{"connection between a person's 'inner world' and 'outer world'"}, where individuals can express, reflect and process thoughts and ideas that come to them. This embodied connection calls for methodologies that focus on bodily movement and action to learn from how people living with dementia configure their participation \citep{morrissey_creative_2015}.

\cite{stenhouse2013dangling} resonate with the work above in engaging people with dementia in designing and creating their own digital stores. The study invited seven people with early-stage dementia to a workshop where they participated in a series of activities such as recording their voices, and approaches to telling a story. Through the participation, Stenhouse reports the necessary facilitation required in supporting people with dementia in the digital storytelling process. While this required facilitators to have knowledge of recording and narrating stories, the priority was building person-centred relationships with the participants to promote a safe space for sharing the stories. Additionally, people with dementia reported confidence in learning skills they felt they had lost through their diagnosis that was achieved through encouragement and support from the workshop group. Last, the authors argue that not only was the research meaningful for people with dementia, but the development of a diverse set of stories of dementia may support the change of perceptions people have of dementia - a particular challenge I describe in the background literature. 

Similarly \cite{lindsay_empathy_2012} adopted participatory design approaches to design for early stages of dementia through a series of workshops. The author's draw on the KITE approach that prioritised fostering an empathetic relationship between people with dementia and designers. The use of empathy is described as an invaluable consideration when working with people with dementia because of a) the need to tailor PD activities on the needs of people with dementia, and b) that care-partners or anyone who does not have dementia should not act as a replacement for the involvement of people with dementia. 

\cite{lindsay_empathy_2012} highlight several complexities through their participatory design approach that researchers should be aware of when working within the dementia space. For instance, while the team aimed to opening up participation through recruitment stages, structured workshops, and forming relationships with participants, the authors share uncertainty in how their analysis of the data \textit{"could, inadvertently, disempower the participants and undermind [their] relationship with [the] participants" (pg.528)}. The authors highlight the construction of the relationship and bond between people with dementia and the designers, has a knock-on effect where people with dementia were hesitant to critique designs in fear of offending one of the designers. Additionally, while caregivers took part in meetings, their involvement was to assist their loved ones in taking part in the research and communication if needed. This initial work highlights that despite the flexibility of PD, to adapt the approach in dementia requires careful consideration of the relational approaches that researchers may want to engage with to provide involvement of people with dementia. Further, the requirement of care partners highlights an ecology of care that may be considered when conducting participatory design work. 

These ethical and methodological complexities of participatory design with people with dementia are echoed by \cite{hendriks_challenges_2014}. While the paper argues that PD has value in design work and can actively include people with dementia, the authors take the opportunity to highlight that they are several challenges that researchers may encounter during PD. These include the burden of PD on the designer that has been underexamined; activities may be too stressful for the person with dementia where they are doing the researcher "a favour"; and power relations can not be mitigated when research is very one sided through designers and researchers learning a lot from the person with dementia and not vice-versa. Last, the authors describe PD methods are difficult to be translated to a variety of stages of dementia. 

While \cite{lindsay_empathy_2012} , and \cite{stenhouse2013dangling} describe the necessary needs to individualise and personalise PD approaches to fit the needs of people with dementia, the workshops they present in the work would be challenging to implement for those at later stages of dementia who may need creative, non-verbal activities. Researchers will also often recruit early and mid stages of dementia for ethical research and ease of working with reasons. This is not to say the work is inaccurate or unethical, more so that PD approaches are difficult to fit the varying needs of different stages of dementia. \cite{bossen2012impediments} points out similar challenges where providing a long-term engagement, a two-way process and tailored PD approaches is not necessarily sufficient. This mirrors \cite{hendriks_challenges_2014} argument that people with dementia are doing the researcher a 'favour' where they get little involvement. Throughout the thesis, what people with dementia get out of taking part in research is a concurrent theme that runs the data chapters.

To describe the PD approaches I have adapted in the thesis, the following subsections describe two types of participatory design that I have explored during the PhD: 
\begin{itemize}
    
\item First, I describe the adaption of interviews as 'walking-interviews' where people with dementia and families take the lead in the direction of the walking route and conversations - seen in chapter four.
\item Second, I describe the ways people with dementia are participating online through Twitter, blogs and online forums that was an essential aspect in the participatory design approach implemented in the hackathon seen in chapter five. 
\end{itemize}

\subsection{Participating in interviews}
\label{PD:Interviews}
Interviews can be framed in multiple ways but will often be structured or semi-structured. While structured interviews provide a limited set of categories and are used to compare and contrast participants answers to the same set of questions, semi-structured offers flexibility from the interview guide where the opportunity to dive into people's experiences, thoughts and ideas can be investigated in a more approachable way. Of course, this has drawbacks where the free-from method is unlikely to provide comparisons between participant responses. Still, for working with researchers, people with dementia, designers and developers, it is an applicable interviewing procedure to explore complex everyday experiences \citep{horton2004qualitative}.

However, while structured and semi-structured approaches have been popular in dementia focused studies, multiple researchers have adapted the process to fit people with dementia better. For instance, \cite{mayer2013lessons} describe how people with dementia may find answering abstract questions difficult, suggesting that interviews may require approval through people close to the person and materials relating to the questions that may elucidate conversation. \cite{suijkerbuijk_active_2019} systematic review highlight the combination of interviews with observations was a popular approach when involving people with dementia in generative or evaluative phases of a study. In many person-centred approaches, researchers will explore the person with dementia's interactions in an everyday setting. However, interviews become a challenge when working with later stages of dementia or those with verbal deficits, and the researcher will prioritise observational data instead. These explorations are often studied in care homes, the individual's home, and activity centres \cite{keady_involving_2007,wallace_enabling_2012-1}. One space that remains overlooked is how people with dementia interact in outdoor areas and how methodologies can provide a sense of agency and fit into the needs of people with dementia. 

One alternative methodology in social research that has gained popularity is walking interviews or 'go-alongs' that provide meaningful engagements between the researcher and participants and deepen an understanding of lived experiences outside. \cite{hein2008mobile} argue that sharing the 'go-along' experience between the researcher and participants provides a richer set of data during the interviewing and observation process. Further, as Foley emphasises paying attention to the non-verbal for those at later stages, \cite{kullberg2017walking} discuss walking interviews provides opportunities to explore the embodied interactions of people with dementia and that the approach puts less pressure on verbal communication. Additionally, by following the person with dementia on the walk provides a sense of agency within the study where the location is on their terms. As suggested by \cite{kullberg2017walking}, a walking interview approach provides a less stressful surrounding where participants may describe what they are seeing, how they are feeling and even the potential for memories and experiences to be triggered by varying triggers from their senses.

With this in mind, using walking interviews to work with families with dementia provides data-informed by experiences in the moments rather than relying on a participant's memory for approaching meaningful experiences. Furthermore, providing an active activity as opposed to a sit-down interview promoted improvement in health and wellbeing. As such, chapter four engages with the use of walking interviews to build connections between myself and the families and provide an enjoyable and fruitful experience in partaking in the research. 

\subsection{Participating through online platforms}
\label{PD:onlinePlatform}
As I described earlier, people with dementia's lived experiences have been shared across research publications, keynotes, publishing books and blogs \citep{bryden_challenging_2020, shakespeare_rights_2019}. These forms of engagement have provided a more accurate public portrayal of dementia that continues to tackle the misrepresentation, stigma and stereotypes of living with dementia \citep{herrmann_systematic_2018}. Additionally, people with dementia have started to broaden their engagement collectively through advocacy groups that stress the model 'nothing about us without us \citep{oldfield2021nothing}. Within these advocate groups, members provide detailed resources for engaging and understanding a diagnosis of dementia; use online chat rooms to receive and share support; and build frameworks and ethics boards to provide researchers with feedback and insight into ways they may want to involve people with dementia \citep{diaries_deep_2020}. 

Over the past five years, particularly with the push of online interaction over the pandemic, people with dementia are engaging with Facebook groups, microblogging on Twitter and using online chat rooms or Zoom to share stories and maintain social interaction \citep{lazar_safe_2019}. For instance, \cite{talbot_how_2020}describe people with dementia using Twitter as a platform to raise awareness, challenge stigma, and take part in online debates on topics of interest. \cite{johnson_older_2019} recent analysis of online forums for older adults describe challenges in ensuring safety, familiarity and decision-making that heighten hesitancy in adopting online technologies. 

To support collaboration and engagement online, we need further consideration for platform users' articulated needs and desires - particularly those within socially complex contexts. Chapter five explores the deployment of an online platform, Ideaboard, to engage with people with dementia and care partners in conversation with designers and developers before the in-person hackathon. As you will discover from reading the findings, this online platform lacked engagement during the study regarding how people with dementia appropriate technology was severely underexamined \citep{lindqvist2018contrasting} . In chapter seven, we return to the challenges faced in our event by investigating how people with dementia envision their participation in online platforms. 
\section{Ethics}
\label{Method:Ethics}
Ethical approval was attained by Newcastle University Ethical Review Board, who reviewed the four individual ethics documents that the following chapters describe in more detail. Across the four studies, audio, video, photography and field notes were found to be the most suitable for data collection seeing that I anonymised the data during the data treatment stage. This stage consisted of anonymising names, blurring faces, transcripts of audio, and redacting specific quotes that may associate with the participant. As I describe in chapter seven, anonymity was an optional process for the designers, developers and people with dementia as there is a growing concern that institutional protection may hinder someone with dementia's individuality and contribution to the work that came out of chapter six findings that interviewed HCI researchers on the ethical challenges in working with people with dementia \citep{hodge_relational_2020}. While this thesis broadens the debate on dementia in HCI by inviting researchers, designers, developers to the conversation, this ethics sub-section will be about the ethical processes I introduced to involve people with dementia within the studies. Each data chapter describes any additional ethical procedures I followed to involve researchers, designers, and developers to attain ethical approval.

Within the area of dementia, ethical consent processes are a contested debate about the best ways to provide continued decision-making with people with dementia. As \cite{dewing_participatory_2007} describe, traditional methods have often excluded the person with dementia by using a spouse or care partner to provide consent, and the consent process would never be revisited throughout the study. Instead, Dewing and others suggest a model where consent processes are a continuing consideration throughout the research project to provide informed flexibility and involvement from people with dementia \citep{dewing_participatory_2007,slaughter2007consent,mckeown_actively_2009}. 

Unfortunately, during the process of attaining ethical approval, myself and the Ethical Review Board were unable to agree on a set of acceptable methods to mitigate risks for informed consent with people with dementia which caused several challenges for the involvement of people at the later stages of dementia. For instance, in chapter four, I invited three families to co-design family day's out to capture meaningful moments through audio, photography, and 360-degree videos consisting of the family being at the beach or a National Trust site. For one of those families, Philip, living with late stages of dementia, could not provide verbal consent on the day of the study. Despite having his two daughters and wife to assist in the decision making, we could not provide an agreement to take part in the study at that very moment. 
Despite not being able to record or take part in the research, I still took the family on the day out as the trip had been planned, and the sole purpose was to provide an enjoyable day out for the family. Throughout the day, Philip's interest in spending time with myself and his family became more apparent, and by the end of the evening, Philip started to take part in group conversations, tickle his daughters' necks, and say to his partner, \textit{"I love you"}. His family had rarely seen these acts from Philip over the past year. With this in mind, an ethic of care procedure that acknowledges capacity is situational, where capacity can be strengthened through relationships proceeding the initial consenting procedure may have provided the opportunity for Philip and his family to be part of the research study \citep{lloyd2004mortality}.

Nevertheless, following the Mental Capacity Act 2005 \citep{oyebode_mental_2005} and drawing on my main concern of duty of care to participants who were involved within this thesis, I provided a series of ways to support and inform people with dementia in the decision-making process. This consisted of receiving training and certificate to conduct Mental Capacity assessments to verify participants could participate; involving family members when necessary to provide additional explanations within the study; interview/workshop debriefs of the previous conversations we had prior to the next interview/workshop; providing time before the study to get to know the participants to build an initial relationship, and asking and implementing any accessible requirements to the studies to accommodate participants throughout the study (e.g. reminder emails about previous conversations prior to the next workshop). 

Additionally, navigating power dynamics within the studies required careful consideration to support people with dementia as equals who can meaningfully contribute through this work. To navigate such complex topics, I followed the works of \cite{foley_struggle_2019,morrissey_value_2017, lazar_critical_2017} who describe the importance of recognition and adapting methods or interactions when necessary. For this to work, when working with people with dementia, I provided spaces that allowed for the person with dementia to lead the conversation or interaction. For instance, in the day's out described in chapter four, I followed the families around, letting the person with dementia lead and make choices of what we do at that very moment. This also expanded into interviews and workshops where I provided a semi-structured interview that would flow depending on the nature of the conversation. Providing this type of agency also required that I would be flexible with my schedule where interviews or activities could be last-minute rescheduled by the person with dementia. Finally, while this was a process of learning and reflection, it was clear the need to guide expectations of what was possible within each study. Navigating these expectations was overlooked in the first study, which caused several frustrations between myself, participants and partnering charity. However, by the end of the thesis, I set out clear boundaries of what would be possible within the confinements of the study and thesis.

\section{Recruitment}
\label{Method:Recruitment}
For this section, I describe recruiting people with dementia through the thesis via partnering up with a local charity and shifting towards online recruitment over the pandemic. Similar to other sections, this focuses on solely people with dementia. Each data chapter details the recruitment processes on an individual level, including how designers, developers, and other stakeholders were involved and recruited.

\subsection{Working with the third sector for recruitment}
\label{Method:ThirdSector}
For the early stages of the thesis, I worked with Silverline Memories, a local registered charity in Newcastle upon Tyne, UK. The charity prides itself on activities for people with dementia, day trips out, celebrations for members' birthdays and other special occasions. During my time with Silverline Memories, I worked closely with the CEO, Sandra, who initially expressed interest in VR experiences in 2016 when she submitted a post on App Movement (designed at Open Lab), a community commissioning platform that enables users to "collaborate, design, and deploy...mobile application". Due to the platform being limited to a set of app templates, the design of a VR app was not possible. Alternatively, given my interest in VR and dementia, I was able to provide an exploration and support into how Silverline Memories members could use VR. 

The relationship and involvement of Silverline Memories span across the first two chapters that primarily focus on the design of VR and media experiences for families with dementia. Working with the charity provided access to a larger group of people with dementia and their care partners than I initially anticipated, as prior work suggests difficulties in recruitment. However, working with a limited geographical area and recruiting through a gatekeeper did cause several issues. 

For instance, sampling was limited to the members of Silverline, indicating I could not be as restrictive with the inclusion criteria for involvement. As such, throughout my PhD, I worked with multiple different people with dementia from all stages of dementia rather than being able to focus on early or advanced stages of dementia. Second, who was involved in the PhD was significantly influenced by Sandra at Silverline. Sandra suggested certain families I should talk to who would be interested in participating rather than offering involvement to all the members. As such, Sandra's insight and commitment to the members did save me time in gaining trust and getting to know all the members at Silverline. However, it limited the study to some extent, in which I was not able to recruit based on needs or backgrounds.  

\subsection{Recruiting online}
\label{Method:RecruitOnline}
In my final study, I could no longer work with Silverline Memories due to the pandemic. While the charity supported various online Zoom activities during COVID, most attendees were care partners. The current members were primarily in the later stages of dementia and rarely interacted with the online sessions. Instead, I branched out to contact multiple dementia advocate groups who had continued to share their stories and raise awareness online. For example, I reached out to the 3 Nations Dementia Working Group. They shared the recruitment poster in their online bulletin and gained interest from several steering group members. From reaching out to several groups and dementia advocates I knew, only five people with dementia participated in the study described in chapter seven. While the stages of dementia range from early to late, all the participants held a conversation and had strong verbal abilities to engage over Zoom meaningfully. In contrast, the first two studies in chapter four orient towards a more embodied understanding of dementia where participation was not limited to verbal interaction.

\subsection{Reflections on recruitment}
As suggested by multiple researchers in dementia, recruitment often comes with multiple challenges, particularly at later stages of dementia, who are often more under-represented than in other stages of dementia. Working with the groups that I did in the North East, participants were predominantly from a working-class background who often had their spouse as a care partner. Meanwhile, the participants I worked in the study described in chapter seven came from mixed backgrounds, such as an accountant, human rights lawyer, and social worker. 

However, one limiting factor in the sampling and in many dementia works is that many groups are further underrepresented, such as the LGBTQ+ community, cultural diversity, young onset dementia, and advanced dementia. In one of my interviews with Howard from chapter seven, he described that while their working group has "started to get people from ethnic, LGBTQ+ backgrounds, it is still far from being done..." As such, while the work has involved a variety of diverse backgrounds, they are groups that are yet to be integrated into research recruitment pools and dementia networks and activity centres. 


\section{Data Collection}
\label{Method:DataCollection}
I collected data through various technologies and techniques throughout the PhD seen in table \ref{tab:OverviewDataCollection}. To involve people with dementia, I adapted several data collection methods to fit the participants' needs better. These adapted approaches are associated with chapters four, five, and seven. As I continued to learn from my successes and mistakes, the data collection evolved throughout the research. While I detail the data collection and treatments in their associated chapters, it is worth presenting an overview of the four phases of data collection, the type of data I collected, and how I prepared the data for analysis. For an example of the official university data management procedure, see appendix \ref{app:DataManagement}.

% Please add the following required packages to your document preamble:
% \usepackage{graphicx}
\begin{table}[htp]
\centering
\caption{Overview of collected data}
\label{tab:OverviewDataCollection}
\resizebox{\columnwidth}{!}{%
\begin{tabular}{l|llllll}
\textbf{} &
  \textbf{Interviews} &
  \textbf{No. Workshops} &
  \textbf{\shortstack{Field Notes \\ (words)}} &
  \textbf{Photography} &
  \textbf{\begin{tabular}[c]{@{}l@{}}\shortstack{Video/Audio \\ (mins)}\end{tabular}} &
  \textbf{\shortstack{WhatsApp \\ (messages)}} \\ \hline \\
\textbf{Study One}   & x                 & 3 avg. 65 mins  & 20,000  & 447 & 660 & x                                      \\
\\ \textbf{Study Two}   & x                 & x                 & 11,355 & 130 & 157 & 270 \\
\\ \textbf{Study Three} & 22 (avg. 50 mins) & x                 & x       & x   & x   & x                                      \\
\\ \textbf{Study Four}  & 11 (avg. 77 mins) & 12 (avg. 63 mins) & x       & x   & x   & x                                     
\end{tabular}%
}
\end{table}

\subsection{Phase One - Auto-ethnography about understanding my role as a researcher}

A significant part of the thesis builds on two studies where I worked closely with Silverline Memories members to explore virtual reality environments. Through these two studies, I began to understand the researcher's role in conducting studies, and the challenges of involving people with dementia that are highly social and politically complex. Furthermore, by working with family members, I realised many of these challenges should be tackled in a more multidimensional fashion by recognising and including multiple narratives about dementia. In that respect, this chapter takes an auto-ethnographic approach to provide a clear account of my background, history, the perspective on dementia, and design approaches that ultimately impact the participants, the setting and the overarching work. As such, this chapter provides insights into designing VR environments for families with dementia and a personal narrative, that recognises the concerns, dilemmas, and the impact of working within sensitive research spaces. 

For this chapter, the auto-ethnographical account draws from the two studies below consisting of: field notes, audio recordings, videos, and photography. 

\textbf{Study one:} In 2017, as part of my undergraduate dissertation, I worked closely with a local dementia cafe called Silverline Memories, which had expressed an interest in virtual reality in dementia. The primary aim of this project was to explore, via collaborative workshops, the type of virtual reality environments and interactions people with dementia may want. Through these interactions with seven participants, a secondary aim was to design personalised VR experiences for couples with dementia to understand how VR could provide aesthetically meaningful experiences. I published this paper at CHI'18 with the co-authors.

\textbf{Study two:} In 2018, through the master's and first year of the PhD, I continued my collaboration with Silverline Memories to explore the opportunities and challenges of designing enriched personalised multimedia experiences with people with dementia and their families. Adopting walking interview approaches to support wellbeing and provide the family with dementia agency in the interview, I designed a set of days out with all three families to create enjoyable or memorable moments, which I then sought to capture and document with audio recording, 360-degree videos, and photography that followed with a series of workshops to consolidate the personalisation and to store of the created moments from their days out. I published this paper at CHI'19 with the co-authors. The combination of moments captured by the families and ideas from the workshops resulted in the first year of the PhD developing a set of' moment boxes' for the families consisting of dioramas; QR codes to access the 360-videos; VR tours of the days out; day out related sensory objects such as seashells, tree bark; and a set of edited photos of the families on their days out.

\subsection{Phase Two - DemVR: A hackathon exploring public engagement with dementia}

As the research evolved, I realised that it takes time for researchers to become aware of the challenges and opportunities within the populations they are working alongside. One area that I felt was underexamined was how inclusive design might function within the context of larger-scale community events. One such larger-scale event that has gained popularity over the previous decade has been open design events, or hackathons. Design events such as hackathons \citep{olesen_what_2021}, design sprints and workshops, involving as they do interdisciplinary teams interested in innovation, have been said to \textit{"offer new opportunities and challenges for cooperative work by affording explicit, predictable, time-bounded spaces for interdependent work and access to new audiences of collaborators"} \citep{filippova_hacking_2017}. Originating within the tech industry as competitive over-night coding events \citep{jones_theres_2015}, hackathons are events where designers, and developers collaborate over a short intensive period (typically a weekend), on software or design projects \citep{nandi_hackathons_2016}. 

In 2019, I set up a design hackathon to invite designers, developers and people with dementia to develop prototypes of VR environments that encourage shared experiences between people with dementia and others. To provide people with dementia and care partners an opportunity to develop ideas, I planned and arranged a six-week pre-hackathon consultation period to be carried out via 1) an online participatory platform, and 2) in-person workshops with people with dementia and their care partners. I designed the two activities to support people with dementia to engage in design activities to illustrate their desired VR shared experiences and bridge the gap between designers, developers and people with dementia on the online platform. However, as I describe in chapter five, the attempts to involve people with dementia and their care partners were significantly limited to one care partner's involvement engaged through our online platform. Throughout the study, I gathered data in four different ways:
\begin{itemize}
    \item Text data from the online participatory platform, consisting of comments from participants and 11 submitted ideas
    \item Video and audio recordings of three keynotes, 15 minute Q\&A with a person with dementia, and teams' presentations from the two-day event.
    \item Images, audio recordings and text data from team WhatsApp groups show the team's design process.
    \item Observational field notes were taken throughout the event, highlighting conversations with teams and facilitators.
\end{itemize}

\subsection{Phase Three - semi-structured interviews with researchers to elucidate the ethical challenges in dementia co-design}

Another thread of interest that came from phase one was better understanding the ethical challenges when working in dementia and HCI, which resonated with recent work by HCI researchers. For instance, academics in the HCI field are reflecting on the ethical challenges they face throughout their research process, with conversations predominantly occurring in venues such as Town Halls and conference workshops at ACM venues. For myself, these conversations would often occur with dementia researchers I met at CHI, DIS, and Dementia Lab - where we would share our frustrations on involving people with dementia, the longevity of technology, and supporting relationships between the researcher and the person with dementia. In 2019, I invited several dementia-HCI academics to collaborate on design ethics in dementia and HCI research, where we would reflect as a community of practice and elucidate broader concerns about ethics in HCI research.

The data consists of interviews with 22 self-identified designers and/or researchers who reported significant experience in working with people with dementia. Each participant was invited to a 45-60 minute interview which consisted of open-ended questions and prompts such as: 1) experience with institutional ethics processes, 2) technological ethics,
3) power relationships, and 4) research impact. Five of the authors conducted the interviews, which were carried out in person where possible, but otherwise carried out over video calls. With 1,100 minutes worth of interview data, UKTranscription transcribed the interviews, and I re-reviewed transcriptions to ensure anonymisation.


\subsection{Phase Four - Dialogical Dementia Design Toolkit: exploring the type of resources for educating designers and developers when designing for and with people with dementia}

During the final phase, I built upon the reflections of the hackathon, where it became apparent that interactions between those with dementia and others outside of the community remain sparse. The final study considers: how toolkits and other creativity support tools foster dialogical engagement between people with dementia and designers and developers? To explore the work area, I invited 11 self-identified designers/developers and five people with dementia to examine the type of resources developers and designers need to design with people with dementia and investigate how people with dementia envision their potential participation within a toolkit.

Due to conducting the study over the pandemic, the workshops and interviews were all online. They required inviting people with dementia who had access to the internet and had a reasonable function of their verbal communication. I collected the workshops and interviews via Zoom/Teams recordings. Group workshops with designers and developers and one-to-one interviews with people with dementia were split across three stages of data collection. Stages one and two focused on gathering data to develop the dementia design toolkit incrementally. For the third stage, I presented the toolkit to participants to gather feedback and critique in its roughly 'final' stage. Zoom was chosen for the interviews as the participants with dementia preferred it. Designers and developers joined a scheduled Teams meeting for the workshops, with additional activities on a shared online whiteboard space (a Miro board).

For preparing the data, I initially used two auto-transcription tools to speed up the transcription phase. This was due to the first two stages of the design process requiring a quick and iterative process for the affinity diagramming, which helped shape the early design rationales and initial toolkit components. For the designer and developer workshops, I used the Microsoft Team's built-in recording and transcription tool that Newcastle University supports. In contrast, given the data sensitivity for the interviews with people with dementia, I used the Google Pixel on-device Audio recorder that supports offline text to speech translation. This was due to privacy issues regarding Zoom and Otter.ai that reuse data to improve their AI model accuracy. Following the auto-transcription process, I revised the transcriptions for any misinterpretations and added any necessary anonymisation. 

\section{Qualitative data analysis}
\label{QualDataAnalysis}
We see the use of qualitative research more regularly in dementia research. The approach allows the researcher to build a complex understanding of people with dementia in their natural setting \citep{mckeown_actively_2009}. One challenge lies in how open-ended qualitative data is as opposed to quantitative. Thematic Analysis (TA), which I primarily used to analyse the collected data, is a popular approach to qualitative data to examine common connections between datasets. TA is the process of \textit{"identifying, analysing, and reporting patterns (themes) within data"} \citep{braun_using_2006}. Given the rich descriptive data I collected through the PhD, TA provides an analysis "that reduces the data in a flexible way that dovetails with other data analysis methods". \citep{kiger2020thematic} suggest that thematic analysis is a proven approach that works well for providing highly descriptive and conceptual findings in their results. As thematic analysis provides steps to organise the data through labels and themes, the process provides transformative approaches to understanding the meanings and experiences of the participants within the data. Further, \cite{braun2012thematic} emphasise that researchers should draw attention to the actual process researchers undertake to ensure confidence in the findings. As such, I describe the TA process I followed below.

The thematic analysis approach I followed was in line with the instructions set out by \cite{braun_one_2020}. This process consists of seven steps: 1) preparing data through transcripts and additional data cleaning; 2) familiarising myself with the data while referring to the research questions. Following, using a whiteboard / Miroboard - order the data to make sense of the similar conversations between participants; 3) move onto the coding process to identify all relevant data by line-by-line coding, tagging and highlighting anything of interest; 4) organise codes into potential linking themes; 5) reviewing themes; 6) discuss the initial themes with supervisors to see if they fit with the original research questions and name the themes. Finally, step 7) is to finalise the analysis by writing and presenting it in the data chapters. Data was collected and analysed chronologically to the chapters that are set out in the thesis. An example can be found in the appendix \ref{app:TA}.

Furthermore, from the social constructionist approach I adopt in this thesis, the epistemology has significant implications for the thematic analysis approach regarding identifying themes that represent meaning or meaningfulness to the topic instead of the recurrence within the data. For instance, \cite{byrne2021worked} draws on an example by Braun and Clarke: 

\begin{quote}
\textit{"…in researching white-collar workers’ experiences of sociality at work, a researcher might interview people about their work environment and start with questions about their typical workday. If most or all reported that they started work at around 9:00 a.m., this would be a pattern in the data, but it would not necessarily be a meaningful or important one."} \citep[pg. 37]{braun2012thematic}
\end{quote}
In this example, \cite{byrne2021worked} is drawing attention to when researchers might require conceptualising meaning from their insight into the coding process to ensure themes are relevant to the research questions.

\subsection{Validity and reliability in research}
\label{TA:Reliability}
One of the biggest challenges qualitative researchers undertake is assuring their research is trustworthy and scientifically rigorous \citep{finlay2006rigour}. These challenges are criticised for ambiguity in the researcher's analytical processes, findings being a collection of opinions subject to research bias, and poor justification of the methods adopted \citep{rolfe2006validity}. Typically, reliability, validity, and generalisability concepts provide a framework for evaluating and conducting quantitative research. However, whether these terms are appropriate for qualitative research is continuously debated among academic researchers, given how inherently different qualitative research is from quantitative \citep{ryan2009rigour}. Alternatively, \cite{noble2015issues} propose a set of strategies and ensure trustworthiness in qualitative research:

\begin{enumerate}
    \item Accounting for personal biases, which may have been influenced by the findings.
    \item Acknowledging biases in sampling and ongoing critical reflection of methods to ensure sufficient depth and relevance of data collection and analysis.
    \item Meticulous record keeping, demonstrating a clear decision trail and ensuring interpretations of data are consistent and transparent.
    \item Establishing a comparison case/seeking out similarities and differences across accounts to ensure different perspectives are represented.
    \item Including rich and thick verbatim descriptions of participants’ accounts to support findings.
    \item Demonstrating clarity in terms of thought processes during data analysis and subsequent interpretations.
    \item Engaging with other researchers to reduce research bias.
    \item Respondent validation: includes inviting participants to comment on the interview transcript and whether the final themes and concepts created adequately reflect the phenomena being investigated.
    \item Data triangulation, whereby different methods and perspectives help produce a more comprehensive set of findings.

\end{enumerate}
Although there is no one-size-fits-all approach to providing trustworthiness in qualitative research, embracing some of the strategies above will provide work that is more mindful of the assumptions and methodological choices made within the researcher's work.  


\section{Reflexivity}
\label{Method:Reflectivity}
Although chapter four acts as a reflective piece that \textit{"acknowledge[s] the multiple roles, identities and positions that researchers and research participants bring to the research process"} \citep[pg.395]{milner2007race}, it is worth examining and reflecting on the way I view the world and the personal understanding of my role as a researcher as the thesis progressed. From early on, it was very apparent that my experiences during and outside the studies would influence the interpretations I made through my analysis. As such, \cite{probst2015eye} describes reflexivity as:
\begin{quote}
\textit{"Despite its "messiness," reflexivity remains a fundamental way, particularly in qualitative studies, to bolster credibility by parsing the research endeavor into its mutually affecting parts and documenting the pathways through which knowledge was generated." \citep{probst2015eye}
}\end{quote}

Reflexivity offers a critical way to stay self-aware through the thesis. Still, more importantly, it is a way to acknowledge the emotional and often challenging experiences that a researcher may encounter when working within sensitive settings - like dementia. \cite{corlett2018reflexivity} recognises reflexivity offers readers insight into the researcher's motivations and interests, which is crucial because research is \textit{"as much the researcher's story as it is the story of [the] participants"}. Furthermore, this reflective practice plays a critical role in the thematic analysis to understand the different roles and identities between the stakeholders and myself. \cite{day2012reflexive} pays further attention to recognising the emotional labour in conducting research and how it is an integral part of research data. Day pulls from Hoffman's work and characterises that emotional labour "shifts the power relationship between interviewer and [participant]" where the researcher and participant produce a shared knowledge about the research. To unpack my reflections, I first describe my positionality, followed by a reflective piece from my field notes highlighting the personal development of becoming a researcher.

\subsection{Positionality}
\label{Method:Positionality}
I received my undergraduate in computer science (CS), where the modules prioritised software engineering skills such as coding best practices, databases, and back-end and front-end web technologies. Until the third year, the course taught very little about Human-Computer Interaction. In the third year dissertation, I went on to work in Open Lab where I explored the role of VR for people with dementia. While the VR interest was due to tech companies' novelty and recent excitement, the dementia thread was from a family experience where my Grandpa lived with dementia. Consequently, as I describe in chapter four, how my Grandma delivered care and described dementia significantly influenced my perspective and expectation of working with dementia. 

Following my undergraduate, I joined Open Lab, Newcastle University as a PhD candidate as part of their centre for doctoral training in Digital Civics. The four-year programme combines a Masters of Research (MRes) in Digital Civics and a three-year funded PhD. As part of the doctoral training, each year, Open Lab enrolled a set of ten students through a cohort approach that consists of diverse interdisciplinary skills and backgrounds. Coming from a computer science background, the design and critical thinking required for HCI was a skillset that was untaught and needed continuous adjustment and learning through reading and integrating the learnings into practice. For example, my early work was more technically focused through an emphasis on 'techno-fixing' where I envisioned technology could provide a solution to societal problems \citep{jongsma2017usual}. However, over time as the thesis demonstrates, the focus shifted toward understanding the ethical complexities of adding technology to an already complex and sensitive setting and how technology might fit the needs of the person with dementia.

\subsection{The researcher's role}
\label{Reflexivity:Questions}
Throughout the PhD, I have made many relationships with participants I still talk to after finishing the study. For instance, Jim, a participant in my final study, will still email or occasionally Zoom to chat about dementia activist topics and general day-to-day conversations. This is often due to my early recruitment stages requiring openness from myself to get to know the participant and make them feel comfortable in my presence. Typically, participants would often ask my \textit{"why are you looking into dementia?"} Looking back at my reflexive journal - that researchers recommend for during the research process and coding stages, I found this excerpt from  replying to the question above:

\begin{quote}
 \textit{   
\textbf{James}: Well, I'm not sure if you remember, but last year [2017] I worked with a couple of you to create some VR experiences that looked at ways to personalise the environments for members here [ at Silverline]. As I got to know the area more, I talked to my Grandma about my research. One of the reasons I thought about dementia research was because of her. My Grandpa had Alzheimer's in his early 50's. He sadly passed away when I was five… Of course, I didn't get to know him that well, but through the circumstances my Grandma shares stories of him. She shared stories of before the diagnosis and many that of her relationship with him after his diagnosis too.}\par


\textit{\textbf{Kate}: "are you close with your Grandma?"
}\par


\textit{\textbf{James}: "Very much so... I call her every few days and we talk about all sorts… To me, It's been interesting to hear my Grandma's side towards caring for my Grandpa too. The stories of her having to learn how to organise the bills, or mortgage – she had to take on so many social roles that he once proudly had… But also, she told me when she told him to get off his back-side and stop feeling sorry for himself… She would make alternations around the house to ensure he could do many of the roles he once felt like he lost – at least to the extent that he thought he could see fit. }
\end{quote}

\textbf{My reflexive journal comment: 
}
\begin{quote}
\textit{"As I began to share stories of my family, Kate started to cry. Kate shared how much the story reflected her experiences too. We started to share stories from each family side. Kate had gone through similarities to my Grandma with having to change significant social and, even in their eyes, gender roles. I feel perhaps Kate and I have bonded over these stories. Maybe she trusts me more now she knows why I care for making some change in dementia? We both cried, laughed and smiled, listening to each other's stories. It felt strange to me that this felt so wrong. A relatively common interaction among friends, maybe not so much between somewhat strangers, but what made me feel 'wrong' was that my openness made me assume I was a bad researcher."}
\end{quote}

Although the reflective texts in my journal are not part of the data I analysed, they helped with the personal development of my role as a researcher. Throughout the thesis, I was aware that by working with people with dementia and having families ties to the condition, the stories participants shared would impact me in ways beyond the academic day-to-day. During my time conducting this research, I distinctly remember questioning 'how is my research helping dementia? and 'what is even the role of me, the researcher, in this work? As I describe in the reflexive journal, at the time, I felt guilty of sharing personal stories and thought I might be contributing to an emotional burden through sharing rather than 'making things better'. However, over time, I realised that emotional work is expected in this type of research, and that is okay. 

At the start of 2020, I received the news that Dorothy, who showed so much enthusiasm and joy for the Shania Twain VR concert hall I created, had sadly passed away. Dorothy and I never verbally communicated, I still remember all these years later the first time she smiled and started to 'hmmm' to 'You're still the one'. I'm not sure if that moment or myself stuck with Dorothy afterwards. However, I learned something significant in that interaction, at the very least.

I learned that part of research is about accepting and reflecting on those emotions and experiences researchers have within this sensitive type of work. It is not as simple as collecting a set of stories and writing an analysis about their lives. Instead, researchers are part of that experience and the experience. While I learned a lot of unique insights into dementia through this PhD, I also have learned many life lessons that influence who I am today. For that, I am very grateful to my participants for willingly sharing their stories and wisdom.

\section{Summary}
\label{Method:summary}
This chapter gives a detailed overview of the research approach I undertook to explore the research questions I set out in the introduction. At the start of this chapter, I describe participatory design and how it has been reconstructed to fit the needs of people with dementia, including the ethical considerations required when working in sensitive settings. From here, I describe ways I have adapted the participatory design to fit the needs of the thesis. I then introduce descriptions of the data collection within each data chapter that are described in more detail in their corresponding chapters. Finally, I conclude with my data analysis approach, thematic analysis, that I used to make sense of each chapter's data supported by reflexivity. In the next chapter, I introduce an auto-ethnography of my initial insights into participatory design work with and for people with dementia, which is a source of motivation for the subsequent studies.
