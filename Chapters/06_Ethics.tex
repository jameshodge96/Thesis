\chapter{Learning from Ethics in Dementia Research}
\label{EthicsChapter}

\section{Introduction}
\label{Ethics:Intro}
In this chapter, I present insights into the ethical experiences and practices in the field of HCI and dementia, where I interviewed 22 researchers from diverse countries, institutions, and disciplines. As the previous chapters have explored the involvement of people with dementia, students (designers/developers), this chapter examines the insights strictly from the researchers perspective of their own cultivated practices. While exploring researchers concerns of participatory approaches in this line of work resonated in chapter four, many conversations I had with dementia researchers at HCI conferences sparked interest in collaborating on this particular project. 

A version of this chapter was published at CHI'20 \citep{hodge_relational_2020} with collaboration from  Dr. Sarah Foley, Dr. Rens Brankaert, Dr. Gail Kenning, Dr. Amanda Lazar, Dr. Jennifer Boger, and Dr. Kellie Morrissey. In this study, I was responsible for methodology, data collection, data analysis, leading the study, and writing the paper. While I directed the study, the paper is highly collaborative and it must be acknowledged that many of the authors influenced the ideas and arguments of the paper. With this in mind, I revisited the study and expanded the discussion section to a) fit with the thesis themes, and b) to ensure the contribution in the thesis is firmly my own.

My conversations with the paper's authors resonate with concurrent conversations predominantly occurring in venues such as Town Halls \citep{munteanu_sigchi_2019,bruckman_cscw_2017,frauenberger_research_2017} and conference workshops at ACMvenues \citep{waycott_ethical_2015,lazar_hcixdementia_2018}. A particular interest has arisen in working with participants in sensitive contexts, due to the unique challenges that arise due to what it means to participate when capacity is difficult to ascertain \citep{foley_printer_2019,lazar_using_2014}, in verbal processes for people who are often not verbal \citep{knapp_nonverbal_2013,kontos_integrating_2018,john_killick_claire_craig_creativity_2012}, and recognition of participants involvement throughout the study \citep{wallace_enabling_2012-1,lindsay_empathy_2012,morrissey_creative_2015}. To date, these conversations have helped share experiences that are often based on a single research project. Yet we are missing an understanding of how researchers in a diverse array of contexts handle ethical decisions. As an example of a topic that emerged in our paper through considering different viewpoints, Ethical Review Boards (ERBs) are in place to ensure research is following standard ethical principles, with the aim of protecting the participants, researchers and research institutions \citep{flicker_ethical_2007}. Despite playing a key role and coming up repeatedly in town halls in terms of questioning how the HCI community can think about ethics given the variation (or lack) of ERBs in the international context, there has been little research to date that examine the impact of ERBs on research in HCI. With this in mind, this chapter takes design ethics in dementia and HCI research as a study to reflect as a community of practice and to elucidate broader concerns about ethics in HCI research.

\section{Related work}
\label{Ethics:RelatedWork}

\section{Methodology}
\label{Ethics:Methodology}

\subsection{Participants and recruitment}
\label{Ethics:Participants}
I recruited 22 self-identified designers and/or researchers (12 women, 10 men). Each participants reported significant experience in working with people with dementia in the design of technologies and services. Participants demographics are summarised in table \ref{Ethics:Demographics}. I approached recruitment through purposive sampling methods


Mindful that we were about to engage with a large number of researchers across several disciplines, and on sensitive topics, we adopted the reflexive position of ‘connected knowing’, as articulated by [7] which recognises disagreement or disparity between viewpoints, but adopts a strategy of empathy instead of judgement or argument. Knowledge, in connected knowing, comes from the ‘inside’ - of a phenomenon, an account, or an experience.

% Please add the following required packages to your document preamble:
% \usepackage{graphicx}
\begin{table}[htp]
\centering
\resizebox{\textwidth}{!}{%
\begin{tabular}{c|cccc}
\textbf{Name} & \textbf{Discipline} & \textbf{Career Stage} & \textbf{Gender} & \textbf{Place of Practice} \\ \hline
Emily & Design & Mid & F & UK \\
Verna & HCI & Early & F & Singapore \\
Neville & Design & Early & M & The Netherlands \\
Louise & Psychology & Early & F & Ireland \\
Sofià & Design & Early & F & The Netherlands \\
Isla & Psychology & Early & F & Ireland \\
Martin & Computer Science & Student & M & United Kingdom \\
Niamh & Speech \& Language & Early & F & United Kingdom \\
Kevin & Computer Science & Senior & M & United Kingdom \\
Lucas & Computer Science & Mid & M & United Kingdom \\
Micheal & Psychology & Senior & M & Ireland \\
Jessica & Informatics & Early & F & USA \\
Daisy & Design & Mid & F & United Kingdom \\
Beth & Social Science & Early & F & United Kingdom \\
Enzio & Design & Early & M & Belgium \\
Mary & Design & Early & F & Belgium \\
Holly & Design & Student & F & The Netherlands \\
Dion & Design & Student & M & The Netherlands \\
Thomas & Computer Science & Junior Developer & M & Sweden \\
Jarod & Computer Science & Junior Developer & M & United Kingdom \\
Lisa & Computer Science & Mid & F & Canada \\
Katie & HCI & Mid & F & Australia
\end{tabular}%
}
\caption{Participant demographics}
\label{Ethics:Demographics}
\end{table}

\subsection{Ethics}
\label{Ethics:Ethics}
Newcastle University granted ethical approval for this study. Each participant was emailed an overview of what would be expected from the interview and the study, and provided with consent form and information sheets. Due to the sensitivity of the topics discussed in the interview, all participants have been anonymised for privacy purposes.

\subsection{Data collection}
\label{Ethics:dataCollection}

\subsection{Data and analysis}
\label{Ethics:Analysis}


\section{Findings}
\label{Ethics:Findings}

\section{Discussion}
\label{Ethics:Discussion}

\section{Summary}
\label{Ethics:Summary}