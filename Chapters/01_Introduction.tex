\chapter{Introduction}

\section{Overview}
\label{Intro: Overiew}
In recent years, Human-Computer Interaction (HCI) projects in sensitive contexts have considered how to design research with people who may be regarded as vulnerable or part of marginalised communities \citep{waycott_challenge_2015}. One of these populations frequently designed \textit{for and with} are people with dementia \citep{suijkerbuijk_active_2019}. Dementia is a neurodegenerative condition that produces varying cognitive changes. Given that people may experience changes to their cognitive abilities, increasing need for care and making judgements, many of the social and cognitive consequences of dementia can be framed as ethical concerns, making it a challenging space for research \citep{herrmann_systematic_2018}.

Moving from an early, medicalised understanding of dementia as a process of decline and memory loss, recent social justice and rights-based responses have resulted in a more holistic understanding of the condition \citep{shakespeare_rights_2019}. This evolving understanding of dementia is mirrored in social and technological responses in research, which moved from early assistive technologies focusing on bridging a ‘cognitive gap’ \citep{mulvenna_supporting_2010} to experience-centred design that fosters creative expressions of personhood \citep{morrissey_value_2017} to more recent work on supporting wider social engagement with people with dementia \citep{foley_care_2019, lazar_safe_2019, welsh_ticket_2018}. By understanding best practices of designing with and for people with dementia, HCI researchers have developed a range of approaches that centre on involving people with dementia at all stages of technology design, as opposed to engaging this population solely at the end of the design process, e.g. by inviting participants for ‘user testing’ or ‘user evaluations’ \citep{brankaert_intersections_2019,schorch_designing_2016, vines_designing_2013}. However, by providing a more relational approach in such work, researchers and participants face different challenges. For instance, \cite{hendriks_challenges_2014} described the challenges of longer-term commitment to projects to support trustworthy relationships between the researcher and participants. Furthermore, the authors drew attention to the implicit decisions designers and researchers may make because they have specific skillsets or expertise.

With the increase in multidisciplinary teams (e.g. encompassing designers, developers, and researchers), researchers have designed multiple approaches and design thinking activities to ensure designers and developers  \textit{``understand the design situation and the problem at hand and to explore and experiment with potential solutions.''} \citep[p.21]{dalsgaard2017instruments}. In the context of dementia and HCI, this has led to inviting students to collaborate in co-design methods with care home residents by developing life story work \citep{foley_student_2020} and storytelling projects \citep{hannan_zeitgeist_2019}. \cite{hendriks_valuing_2018} further supported the importance of designers and students building a relationship with the people we are designing for and within the context of dementia. The authors argue that design decisions \textit{``emerge from the relationships designers build''} \citep[pg. 3]{hendriks_valuing_2018}. However, while opportunities to work in more non-traditional settings such as care homes may be possible through university classes, these are often limited to a small selected group of students or courses focusing on healthcare and psychology \citep{kinnunen_understanding_2018}, meaning that those who are taught technical or design disciplines at university miss out on opportunities to gain experience with the vulnerable populations for whom they may end up building.

This thesis address the following key aim: \textit{``To understand how participation might be configured for people with dementia to shape the design process of technology?''} Previous HCI research working closely with people with dementia stresses the need for method adaptability to ensure inclusive and meaningful engagement. In turn, this raises the question of what approaches should be considered by designers and developers that would allow them to design collaboratively with people with dementia. This thesis describes four studies involving the perspectives of diverse stakeholders who are commonly part of the design process when designing for and with people with dementia. Each study examines the competing interests and expectations of involving a diverse group of stakeholders to collaborate in the design process. These studies' findings and discussion sections contribute new knowledge concerning how participation is adapted and shaped to develop inclusive design spaces that support mutuality in creating new technologies and systems.

\section{Research background}
\label{Intro: ResearchContext}
To provide an understanding of technology design in the context of dementia, the following section is split into: a) continuing to experience the world with dementia; b) technology and dementia; and c) learning through design. 

\subsection{Continuing to experience the world with dementia}
\label{Context:Dementia}
Historically, research on dementia has often emphasised its biomedical origins \citep{lyman_bringing_1989}. As dementia progresses, it often creates conflict between the person and their surroundings, as the surroundings can become unfamiliar, and this can also cause difficulties with coexisting with others \citep{langdon_making_2007}. People with dementia can find themselves feeling stigmatised within new roles of `patient', `demented', and `in need of care' \citep{cohen-mansfield_utilization_2006}. This can deprive the person with dementia of their personhood and change their quality of life \citep{lawrence_improving_2012}. The progression of dementia means that the individual's role within the family structure can change, as they become the care-receiver, and therefore, the impact of dementia can be troubling for both parties \citep{dupuis_moving_2012}. In particular, social activity may decrease, which entails several `knock-on' effects, such as a decline in emotional well-being, and increased social isolation and depression \citep{bartlett_citizenship_2014}. 

In the 1990's, \cite{kitwood_towards_1992} promoted the concept of personhood in care, which proposes a series of person-centred approaches to acknowledge the entire individual to ensure that they are heard, included and understood. Since then, many researchers have built upon and used person-centred approaches to dementia care and have called attention to how we communicate with people with dementia \citep{oyebode_mental_2005}, debated the need for ongoing consent processes \citep{dewing_participatory_2007} and promoted the need to attune to embodied, non-verbal communication \citep{group_patron_2019, twigg_dress_2013} as key considerations in ensuring the person with dementia is respected and engaged within their own care. These practices, largely initiated within nursing and social care, have implications for research and design that seek to work with and for people with dementia, and to avoid practices that devalue or disregard their experiences. \cite{john_killick_claire_craig_creativity_2012} highlight the use of arts and creativity to provide alternatives to verbal communication that demonstrate unique ways for people with dementia to share their experiences. Through humour, dancing, acting, music, movement and fashion, people with dementia can communicate and share experiences that move beyond verbal communication.

While recognising that the person with dementia's individuality has raised awareness in respecting the needs, wants and desires of the person with dementia, \cite{bartlett_personhood_2007} suggested that embedding person-centred values within a citizenship lens integrates a more socio-political and critical understanding of dementia. In particular, the lens acknowledges the challenges of stigma, relationship dynamics and the unique nature of the individual with dementia. Moving towards a citizenship model has empowered people with dementia to share their experiences, which has impacted policymaking, practice and research \citep{weetch_involvement_2020}. For instance, \cite{bryden_challenging_2020}, a pioneering dementia advocate, has written many blog posts, presentations and personal books advocating for changes in media and public portrayals of dementia in an attempt to change dominant misconceptions about, and stereotypes of, the condition. The sharing of experiences has an impact in two ways. First, advocating and sharing experiences fosters purpose and impact in understanding dementia on a more inclusive level; and second, the stories promote awareness and improve public perception \citep{reynolds2017stigma}.

To an extent, improving the public perception of dementia and the citizenship approach echoes policy and practice. In 2020, the UK set out the Prime Minister’s challenge to make the country the best place to live with dementia, and be the leading place in the world for dementia research \citep{budgett2021designing}. This challenge set out to tackle dementia-friendly health and care settings; educate people earlier about the risks of developing dementia; and provide more opportunities for people with dementia to partake in research and present talks about their experiences. \cite{keady2017social} argued that these aims by the UK present dementia as \textit{``everyone’s business and responsibility, from architects to town planners, public transport providers to shop owners, next-door neighbours on the street where the person with dementia lives - the list is seemingly endless''}. As such, given that dementia is part of a complex ecology of care, which considers the person with dementia, friends, family, healthcare systems, researchers, designers and everyday interactions, it is particularly timely to unpack what it means to design methodologies that consider the diverse stakeholders and infrastructures that surround and often hold up our work. 

\subsection{Technology and dementia}
\label{Context:Design}
Initially, technology design and dementia focused on cognitive decline, monitoring and management. This resonates with the biomedical approach to care, which concerns the neurocognitive condition and decline in abilities \citep{o2022conceptualizing}. This early work in technology design considered using sensors, surveillance technology, and devices to compensate for the physical and cognitive deficits of people with dementia. \cite{bharucha2009intelligent} reported that these early studies would be often be guided by family and professional care staff instead of people with dementia. In turn, this neglects the needs of people with dementia, who are the end users. For instance, \cite{astell2006technology} reported that the ability of care partners or staff to track and follow the person with dementia may devalue their agency. This becomes particularly becomes challenging when monitoring technology may reduce human contact between the person with dementia and the care partner. In this way, \cite{astell2006technology} argued that these ethical issues in early work in the area of HCI and dementia largely stem from the lack of involvement of people with dementia in the design process to understand their needs and interest in technology.

Such lack comes from the challenges of verbal communication \citep{majlesi2017video}, the potential stress research may cause and researchers being unaware of such biases, which may assume that people with dementia lack the capability to contribute to the design processes \citep{manthorpe_person-centered_2016}. As HCI began to extend person-centred approaches to dementia, studies such as CIRCA \citep{astell_stimulating_2010} and KITE \citep{robinson2009keeping} challenged the role that people with dementia play in the design process of technology. Through these initial studies, researchers adapted their methodologies to centre the voices of people with dementia, providing the tailoring of technology to the user's assistive needs. 

From here, HCI research that extends from person-centred approaches to dementia is a growing body of work that relies on relational processes as the basis of design. This has resulted in the introduction of technologies that evoke emotion \citep{wallace_enabling_2012-1,houben_foregrounding_2019,dixon_approach_2020}, engage in creativity \citep{lindsay_empathy_2012,morrissey_im_2016} and support inclusion \citep{welsh_ticket_2018,foley_printer_2019,treadaway_sensor_2016}. For instance, \cite{wallace_design-led_2013} used a tailored approach that centred the importance of personhood by paying attention to a people’s individuality to design bespoke digital artefacts for the participants. Similarly, \cite{lindsay_empathy_2012} described the need for more interpretative data approaches for those in the later stages of dementia, which in turn, may require longer-term projects and relationships to form throughout a study. 

More recent work has seen a critical turn to understanding personhood even with those in the later stages of dementia, where non-verbal and ambiguous interactions might be present \citep{lazar_critical_2017}. \cite{treadaway_sensor_2016} emphasised that recognition and appreciation are crucial even when leaning on tacit, creative activities to support non-verbal interactions. This critical turn is further supported in work by \cite{morrissey_value_2017}, which took an experience-centred approach that shifts the way we see people with dementia-related cognitive deficits as contributing to design choices \citep{wright_aesthetics_2008}. While prior work may have focused on alleviating a person's cognitive deficit, the critical perspective widens our approach to inclusivity, by celebrating what a person has to offer through more creative and engaging approaches, where those living with a broad spectrum of dementia-related changes can also participate \citep{foley_printer_2019}.

\subsection{Learning through design}
\label{Dementia-Design}
As HCI and dementia research has developed, this has required multidisciplinary teams to develop and design technology. In HCI, researchers have considered novel approaches to undergraduate education to provide designers and developers with the skillsets and ethics to make more sensitive design choices. Researchers have often developed toolkits and other creativity support mechanisms to support empathy-based skill development for individuals implicated in the design process. For instance, \cite{chen2020interaction} introduced a toolkit to turn conference papers into actionable guidelines and design tips through the curation activity of undergraduate researchers, who review papers and summarise the key findings down into a series of guidelines. However, the use of design toolkits may face challenges where other cultural contexts and settings interact \citep{peters2020toolkits}, such as when values, goals and technologies displayed on cards may have little to no meaning within the setting.   
It is clear that creating specialised toolkits for use in particular settings or contexts requires specialist input to ensure the content is appropriate \citep{alshehri2020scenario,meissner2018schnittmuster}. For example, \cite{craig2021development} developed an ethical roadmap for design at end-of-life; this roadmap comprised components such as `informedness’ of consent questions, provocations and value cards to support developers and designers to reflect on and question their approaches to this very contested area of work. To curate the cards and content for the ethical roadmap, the authors worked with a diverse team of experts in end-of-life matters, digital media and physical object design, as well as with participants whose experiences in bereavement spoke to the toolkit’s use. In a similar study, \cite{shinohara2020design} described a series of design cycles requiring various stakeholders to develop a set of method cards for social accessibility. While \cite{shinohara2020design} emphasised that the toolkit should not replace students directly speaking to users, they explained that the toolkit provides \textit{``information about how to interact with expert users''} and that this \textit{``helped students to know how to start conversations and guide them toward productive discussions''}. In turn, this raises the question as to what sorts of toolkit interactions we should provide to designers and developers that allow them to collaboratively design with people with dementia while at the same time, prioritising speaking and involving people with dementia. Having framed these considerations about design for dementia, from three perspectives, I now turn to consider the research aim and questions that this raises.

\section{Research aim and questions}
\label{Intro:RQ}
The key aim is:
\begin{quote}
    \textit{``To understand how participation might be configured for people with dementia to shape the design process of technology?''}
\end{quote}
To tackle the research aim, I provide three research questions that I explore throughout the thesis. The following sub-sections introduce these three questions.

\subsection{Research question one}
\label{RQ1}
\begin{quote}
\textit{``How can we use participatory design approaches to provide meaningful and engaging experiences for people with dementia?''}
\end{quote}
Chapter four presents novel methods, such as days out and workshops that support people with dementia to talk about what is important to them and how they might use virtual reality (VR). In chapter five, the analysis of the hackathon demonstrates the challenges of participation when the event may not be aligned with the needs or interests of the end users. The discussion raises insights into ways of improving recruitment to involve people with dementia and steps to promote more inclusive, community-driven events. Finally, chapter seven explores the resources developers and designers need to design with people with dementia and investigates how people with dementia envision their contribution to the set of resources. In this chapter, the analysis and discussion provide an understanding of how people with dementia want to share the `designer' role and that incentives are necessary for participation and engagement. This exploration draws attention to how we speak about dementia, to what extent people with dementia want to contribute to technology design, and ways to ensure that the person with dementia is respected and engaged in research.

\subsection{Research question two}
\label{RQ2}
\begin{quote}
\textit{``What are the ethical implications for people with dementia to participate in HCI research?''}
\end{quote}
Given that people with dementia may experience changes to their ability to problem solve, make judgements and find that their need for care is increased, many of the social and cognitive consequences of living with dementia may make it challenging to participate and be involved in research. In chapters four and five, I reflect on the ethical challenges of knowing when and how to involve participants in the research, and appropriately acknowledging participants' contributions to the work. From these findings, in chapter six, 22 dementia and HCI researchers are interviewed to present insights into the careful ethical considerations required when working with people with dementia and their families.

\subsection{Research question three}
\label{RQ3}
\begin{quote}
\textit{``What are the competing interests and expectations in supporting meaningful dialogue in dementia design research when involving multiple stakeholders - such as people with dementia, developers, designers and researchers?''}
\end{quote}
For the final research question, the thesis examines the types of interests and needs that each stakeholder requires in designing for and with people with dementia. Each chapter builds on answering this research question by exploring the interactions between people with dementia and the different stakeholders. For instance, these four chapters involve the following diverse set of stakeholders:
\begin{itemize}
\item Chapter four: people with dementia, care partners, family members, and friends who engaged in walking interviews and a set of workshops.
\item Chapter five: care partners, designers, developers, and undergraduate students who partook in a hackathon and an online pre-engage phase to increase engagement of those who could not come to the hackathon.
\item Chapter six: designers and researchers who engaged in one-to-one semi-structured interviews.
\item Chapter seven: designers, developers and people with dementia who engaged in a series of workshops and interviews.
\end{itemize}

By working with a diverse set of stakeholders, each chapter builds knowledge on how to balance and respect each stakeholder's different interests and expectations to move towards designing more inclusive and support spaces where people with dementia can contribute to co-creating new technologies and systems.

\section{Thesis structure}
\label{Intro: Thesis structure}
To understand how this dissertation addresses these research questions and the overarching topic, the following section outlines the structure of the rest of the thesis.

\subsection{Chapter two - Background literature}
\label{Intro:ChapterTwo}
This chapter describes prior work on the representation and involvement of people with dementia in technology design and development. Following a review of HCI work in the context of dementia, I highlight three knowledge gaps. The first is that HCI work has primarily focused on mild to moderate stages of dementia as regards participation, resulting in an unrepresentative perspective when involving people with dementia and stakeholders in technology design. Second, I examine the ways in which researchers recruit, involve and represent people with dementia which concludes with how might we support the teaching of dementia awareness for technologists who are the ones making design decisions on behalf of people with dementia. Finally, to conclude this chapter, I describe four areas of interest to this thesis that shape the thesis. The four areas are the following: 
\begin{itemize}
    \item Attending to mutual collaborative relationships
    \item Representation of technology and dementia from a designer/developer perspective
    \item Ethical practice in HCI in the context of dementia
    \item Promoting learning around dementia and technology
\end{itemize}

\subsection{Chapter three - Methodology}
\label{Intro:ChapterThree}
In the methodology chapter, I explain and justify the research approach taken in the thesis that shapes the data chapters. First, I introduce the epistemological approach, a social constructionist approach in which knowledge creation is a collaborative process. Next, I unpack participatory methods in dementia and HCI that emphasise the need to adopt approaches to fit the needs of people with dementia. I then describe how I adapt co-design and participatory sesign methods to fit those needs. The chapter presents discussions regarding the ethical challenges of the thesis; ways in which I captured data; recruitment and location details; and the qualitative data analysis method, thematic analysis. To conclude the methodology chapter, I describe the process of reflexivity, validity and reliability to support the data analysis process. 
\subsection{Chapter four - Sharing a Virtual World with People with Dementia: A Reflective Account}
\label{Intro:ChapterFour}
The chapter is a reflective account of two studies working with a dementia café to design VR and media experiences with and for families with dementia. I worked closely with a dementia café in Newcastle called Silverline Memories, which provides activities and organises celebrations for members’ birthdays and other special occasions. By working closely with families who have members with dementia, the account examines how I adapted participatory approaches to involve people with dementia in more sensitive and meaningful ways. For instance, given that some members of the families were rarely verbal, I used walking interviews \citep{kullberg2017walking} as an opportunity to observe their interactions and listen to their stories if they chose to share. 

By working with family members as well as the person with dementia, the chapter recognises that the development of technology for people with dementia must consider the needs and interests of friends, family and care staff, which make up what is known as the ecology of care. In that respect, this chapter takes a reflective approach to provide a clear account of my background, history, my perspective on dementia and design approaches that affect the participants, the setting and the overarching work. As such, this chapter provides insights into designing VR environments for families with dementia within a personal narrative that recognises the concerns, dilemmas and impact of working within sensitive research spaces \citep{england_getting_1994}.
\subsection{Chapter five - DemVR: Exploring Shifting Sensitivities in a Hackathon for Dementia}
\label{Intro:ChapterFive}
Following chapter four, given that I was required to work closely with participants to design a bespoke set of VR experiences, I was curious about how designing for people with dementia might function within the context of larger-scale community events, which might reduce the time it took to develop those VR experiences on my own. At the same time, local authorities were interested in extending this work to larger groups of people through public-facing events - such as a hackathon. 

The event consists of two stages: a six-week engagement phase to support participants in proposing and refining initial ideas online; and a two-day hackathon inviting designers and domain experts to develop their ideas further. While the event gained reasonable interest from designers, developers and students throughout both phases, the representation of people with dementia and their care partners was limited. The chapter examines the structure of the event and the role this played in the struggle to involve people with dementia and their care partners. The data analysis presents insights into participants’ motivations, design approaches to accommodate the absent user and the design ideas that the teams developed to address the social context of the user. The discussion provides a set of commitments that offer insights into how we might mitigate stereotypes in constructing the end user; ways to improve recruitment for marginalised populations in events; and steps to promote more inclusive, community-driven events. 

\subsection{Chapter six - Learning from Ethics in Dementia Research}
\label{Intro:ChapterSix}
In chapters four and five, I described how engaging in participatory research in HCI raises numerous ethical challenges such as research relationships, participant recognition, recruitment and consent. Undertaking HCI work in sensitive settings amplifies these issues, and researchers in this area have modelled approaches to ensure that participants are meaningfully engaged and represented in their work. In response to this need, HCI researchers are reflecting on the ethical challenges they face throughout their research process \citep{vines_designing_2013}, with conversations often occurring in venues such as town halls \citep{munteanu_sigchi_2019,bruckman_cscw_2017} and conference workshops at ACM venues \citep{davis_ethical_2015,waycott_challenge_2015}.

In this chapter, I take design ethics in dementia and HCI research to elucidate broader concerns about ethics and HCI when carrying out design research with people with dementia and their care partners. I interviewed 22 researchers from diverse countries and institutions who work in dementia design research. The analysis examines the potential challenges researchers face with Ethical Review Boards (ERBs). While prioritising the protection of human subjects, they can inadvertently prevent the full inclusion of people with dementia in research. Researchers also shared insights from their own cultivated practices when establishing clear expectations for participants, knowing when and how to involve participants in the research, and appropriately acknowledging the contribution to our work that participants make. I proceed from the findings to emphasise a set of directions for researchers and ERBs to improve their practices by moving towards more participant-led research, re-framing impact and aiming for research clarity.

\subsection{Chapter seven - Co-creating a Digital Toolkit to Support Design for Dementia}
\label{Intro:ChapterSeven}
From the previous two chapters, it was apparent that: 
\begin{itemize}
\item Representing people with dementia may be complex in public events and causes multiple knock-on effects on design outputs, such as overlooking care partners' interests and teams designing VR environments based on one person with dementia's Q\&A session.

\item Based on the challenges that researchers in chapter six highlighted regarding navigating ERBs and acknowledging people with dementia's contribution, future dementia research should aim to be participant-led. Further, the chapter discusses forming tools or processes to promote conversations between people with dementia and stakeholders commonly implicated in design processes to provide such participant-led approaches.
\end{itemize}

The two chapters question how we can support collaboration and engagement between those being designed for and those doing the designing in the context of dementia. To explore the needs and desires of stakeholders, this chapter presents the design of the Dialogical Dementia Design (D3) toolkit, a set of resources to support co-designing with people with dementia. I invited 11 developers/designers and five people with dementia for a series of interactive workshops and interviews that explored resources needed by developers and designers to design with people with dementia and investigate how people with dementia envision their potential participation with toolkits. The analysis raises questions about the challenges of co-creation through safety and privacy, the sharing of the ‘designer’ role between the different stakeholders, and finally, the type of incentives required for participation and engagement in curating a toolkit. Finally, this chapter concludes with insights into highlighting how we might balance participants' privacy, safety and due recognition; priorities in growing a community-owned toolkit; and the accountability and responsibility designers and developers carry in adapting their working practices for designing within sensitive areas.
\subsection{Chapter eight - Discussion and Future Work}
\label{Intro:ChapterEight}
In the closing chapter, I synthesise the findings from the data chapters and revisit the research questions I set out at the start of this thesis. The discussion aims to answer how researchers adapt methodologies to provide more engaging experiences for people with dementia; how, as researchers, we might mitigate some of the ethical implications for when people with dementia participating HCI research; and how we can balance and design for the competing interests and expectations of diverse stakeholders to ensure the support of meaningful dialogue between people with dementia, designers and developers. 

The chapter concludes by offering a set of components for researchers to balance stakeholder interest throughout the development and running of a study. These three components consider a) an inquiry panel of expert stakeholders to guide and shape research agendas and provide support during the running of a study; b) three critical areas of impact that researchers must consider when starting a project; and c) a set of three considerations to articulate stakeholder interests and priorities through the research process. Regarding the three components, the approach aims to move towards more inclusive design spaces supporting mutuality in creating new technologies and systems.


\newpage
\subsection{Thesis map}
For chapters four to seven, I present a map showing which research questions each data chapter tackles (see figure \ref{fig:RQ_and_Chapters}).

\label{Intro:Thesis Map}
\begin{figure}[htp]
\centering
\includegraphics[width=.8\linewidth]{Images/Thesis_Narrative/RQ_and_Chapters.png}
\caption{Thesis map showing the relationships between data chapters to research questions}
\label{fig:RQ_and_Chapters}
\end{figure}


\section{Contributions}
\label{Intro:Contribution}
In addition to the thesis resulting in a number of publications, I offer three contributions to knowledge in dementia and HCI:

\begin{itemize}
    \item \textbf{Empirical contribution}: This thesis uses data collected through walking interviews, workshops, a hackathon event, semi-structured interviews and a three-stage iterative design process to provide qualitative insights into the experiences and perspectives of various stakeholders implicated in design processes when designing with and for people with dementia. By characterising the motivations and challenges that stakeholders faced when participating in the design of technology, insights gained across the thesis reimagine the role of participation between people with dementia and stakeholders to promote conversations and the inclusion of stakeholders in the technology design process. Each study offers practical examples of ways to create spaces to facilitate and support actively understanding and representing the contributions of people with dementia.

    
    \item \textbf{Artefact contribution}: Within the thesis, two chapters present novel systems and prototypes to facilitate new insights and consider ways for people with dementia to engage with technology and contribute to the design process. At the time of conducting the studies seen in chapter four, the design of VR environments explored how people with dementia might use this novel technology. Through this work, the prototypes provided suggested ways in which technology might provide `in the moment' experiences as opposed to reminiscence methods, which promote recalling memories and tend to draw on stability in long-term memory.

    In chapter seven, I collaborated with designers, developers and people with dementia to develop a lo-fi prototype toolkit to support designers and developers in co-design with people with dementia. The main contribution of this is a series of directions for HCI and dementia research, highlighting how we might balance participants’ privacy, safety and recognition; ways for stakeholders to contribute to and grow a community-owned toolkit; and the accountability and responsibility that designers and developers take when designing in sensitive and ever-changing situations.

    
    \item \textbf{Methodological contribution}:  The thesis contributes a methodological understanding to support an equal relationship between those designing and those being designed for. In chapter four, by inviting families to a set of walking interviews (also known as days out), the families took the lead on the walking route, which provided the possibility of people with dementia directing the flow of the day and expressing when they wanted to share their experiences and thoughts.

    Chapter five reports the methodological challenges of recruiting people with dementia and care partners through online engagement platforms. The discussion highlights how researchers must consider the platforms participants use to communicate online. Additionally, this includes the importance of offline engagement for those without technical abilities. As such, reflecting on how the approach led to a lack of involvement by people with dementia, this chapter suggests ways of improving recruitment to involve marginalised populations in hackathons.
\end{itemize}
