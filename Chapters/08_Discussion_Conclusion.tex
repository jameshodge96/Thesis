\chapter{Discussion and Future Work}
\label{Discussion}

\section{Meaningful participation in design approach}
\label{Discussion:Design}
The meaningful participation design approach draws on the insights from the previous data chapters to meaningfully engage and broaden the dementia debate through a more inclusive approach. The approach consists of three core components that surround a research study. By considering the three components, the aim is for researchers to critically reflect and consider ways to improve stakeholder engagement, ethically engage with research impact, and ensure research designs are rooted in participant-led agendas. The three components are the following:

\begin{itemize}
    \item \textbf{Inquiry Panel}: Developing a group of experts (primarily stakeholders) who guide and shape the research and provide answers to any queries by the research team.
    \item \textbf{Impact}: Consists of three critical areas that researchers should consider when designing engaging and impactful outputs for their studies.
    \item \textbf{Stakeholder considerations}: A set of four considerations to move towards more inclusive and engaging research that articulates the interests and priorities of the stakeholders the research will impact.
\end{itemize}

\subsection{Inquiry panel}
\label{Inquiry Panel}
As described throughout the PhD, a key concern I raise is participants' lack of influence on research agendas. Similarly, \cite{suijkerbuijk_active_2019} highlight that people with dementia are often underrepresented in pre-design and generative phases of technology development. By ensuring research designs are rooted in participant-led agendas, they can contribute to a more ethical, engaged research study. As \cite{dupuis_moving_2012} argue, \textit{"listening and hearing the perspectives of persons with dementia is not enough. We must actively involve them in decision-making to the fullest of their abilities and support their involvement using whatever means necessary" (p.433)}. One common approach that has gained recent attraction in the last few years is the development of dementia activist groups which aim to promote the voices of people with dementia and guide and shape the design of research approaches \citep{weetch2021involvement}. Working with activist groups contextualises a deeper understanding of what work is sensitive and what is not. It also articulates the interests and priorities of the individuals the research will impact. As such, part of the meaningful participation design approach builds upon these activist groups by suggesting that researchers might implement an Inquiry panel in their research.

The Inquiry panel consists of people who have expertise in the domain the research is looking into. For instance, this thesis inquiry panel would involve people with dementia, care partners, developers, designers, and researchers. From here, the research team actively involves the inquiry panel in designing research agendas, feedback on ethical and methodological approaches, and supporting recruitment. By building a panel of experts to support the research team's ideas and approaches, the panel echoes prior work that supports confidence building, empowerment, and providing opportunities for change and development.

Furthermore, a more diverse group of experts within the inquiry panel might enable the building of reciprocal relationships with other members by learning and sharing their expertise and experiences. However, it is worth noting that building a diverse set of voices and backgrounds will require careful consideration and facilitation to ensure voices are represented. For instance, \cite{innes2021s}, who worked with The Dementia Associate Panel (DAP), a group of people with dementia who advocate for change and policymaking, described the panel's ongoing flexibility and adaption to facilitation to ensure meaningful participation. The author used \textit{"self-reported questionnaires at the end of each [panel] meeting...which allowed [the members] to indicate whether the meeting effectively supported their contribution of view and ideas"}. While this thesis describes the need for broadening stakeholders to be involved in the entire research process, it will need the creation of tools and approaches to promote conversation and inclusion of community members, researchers, and other infrastructures that uphold our work i.e., ethical review boards and grant panels. 

To conclude this section, I suggest a set of steps that may be useful for researchers who want to start and develop an inquiry panel:

\begin{itemize}
    \item List the potential groups/people that have expertise on a specific topic (people with dementia, family members, researchers).
    \item Invite these experts to take part in the panel. You may find these people through other advocacy/network groups.
    \item To facilitate inclusive meetings, ask the panel members a) if they understand the project/panel and b) if they require any additional support or practicalities to make their involvement easier, e.g., invitation, signage, refreshments if in person.
    \item At the first meeting, as the researcher, prepare a presentation that describes your plans for the potential project, how long it will be, and the potential outputs/expectations. Following, open the conversation up to the panel to discuss the issues and interests of the project.
\end{itemize}

