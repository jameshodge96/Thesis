\chapter{DemVR: Public Engaged Hackathon}
\label{DemVR: Public Engaged Hackathon}

\section{Introduction}
\label{sec:DemVRIntroduction}


\section{Event context}
\label{sec:ContextEvent}

Motivated by prior work, the authors set out to explore how hackathons could provide a creative and inclusive space for the public to engage on the topic of dementia. This project is part of an ongoing long-term study focusing on the inclusive design of evocative VR experiences for families living with dementia. While our previous work extended concepts from experience-centered (ECD) [4, 66, 70] which require working closely with participants, often one-on-one [102], requests for collaboration from both local dementia community groups and local authorities often saw us trying to extend this work to larger groups of people. At the same time, work in digital civics for social care [19, 76] as well as interest from local authorities prompted us to explore how inclusive design work might function within the context of larger-scale community events had been under-examined. 

\subsection{}

\section{Learning from the event}
\label{sec:LearningEvent}

\section{Considerations for future events}
\label{sec:ConsdierationsEvent}

\section{Conclusion}
\label{sec:Conclusion}